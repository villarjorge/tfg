\section{Objetivos}

\subsection{Objetivos generales}

El objetivo del presente trabajo final de grado es exponer y obtener resultados en la propagación de ondas electromagnéticas en un medio turbulento, tanto teóricamente como a través del método numérico FDTD. 

\subsection{Objetivos específicos}

En cuanto a objetivos específicos podemos mencionar los siguientes:

\begin{itemize}
    \item Realizar una revisión bibliográfica y teórica tanto sobre las bases de radiación electromagnética y mecánica de fluidos como los principales problemas del campo y las formas en las que se resuelven. 
    \item Uso de herramientas matemáticas avanzadas, especialmente aquellas que no han aparecido en el temario de la carrera, como por ejemplo transformaciones de Mellin y funciones hipergeométicas.
    \item Implementación de los métodos numéricos expuestos o en su defecto una exploración del software más utilizado en el circulo académico.
    \item Contraste entre los resultados de métodos numéricos y métodos analíticos. 
    \item Utilizar los resultados obtenidos para informar análisis consiguiente, principalmente en el ámbito del análisis de la señal e imagen. 
\end{itemize}

\subsection{Beneficios del proyecto}

El proyecto implica una revisión bibliográfica y teórica de la radiación electromagnética y la mecánica de fluidos, así como la aplicación de herramientas matemáticas avanzadas principalmente métodos perturbativos y espectrales. Esto contribuye a una comprensión más profunda de los fenómenos físicos subyacentes a la propagación de ondas en medios complejos.

El uso del método FDTD (Diferencia Finita en el Dominio del Tiempo) a través de una librería de Python para simular la propagación de ondas en turbulencia y el contraste de estos resultados con métodos analíticos, valida una potente herramienta numérica para abordar este tipo de problemas. Este desarrollo es valioso dada la versatilidad y el poder del método FDTD en el campo de la simulación electromagnética.