\section{Introducción}

La propagación de ondas electromagnéticas (EM) a través de medios turbulentos plantea un reto polifacético en diversas disciplinas científicas y de ingeniería. Históricamente, las observaciones de los fenómenos inducidos por la turbulencia han abarcado desde el desenfoque macroscópico de los objetos celestes hasta la sutil degradación de las señales de comunicación. Comprender estos efectos es primordial para optimizar el rendimiento de los sistemas, sobre todo en aplicaciones en las que el medio de propagación es intrínsecamente dinámico y heterogéneo.

En el ámbito de la óptica, la manifestación más emblemática y observada de la turbulencia es la visión astronómica (\cite{tubbs_lucky_2003}). Desde la antigüedad, el parpadeo de las estrellas y la claridad variable de las imágenes telescópicas se han atribuido a irregularidades atmosféricas (\cite{noauthor_mystery_1994}). Ya en 1665, Robert Hooke especuló sobre el papel de la variación del índice de refracción de la atmósfera. Más tarde, a finales del siglo XIX y principios del XX, la turbulencia atmosférica fue célebremente implicada en la interpretación errónea de las características marcianas como «canales», poniendo de relieve su profundo impacto en la percepción visual y la observación científica. La percepción de estos «canales» en Marte por parte de astrónomos como Giovanni Schiaparelli y Percival Lowell se atribuye ahora en parte a ilusiones ópticas exacerbadas por las turbulencias atmosféricas. Del mismo modo, el fenómeno común de la neblina térmica sobre superficies calientes y los espejismos más complejos (incluido el legendario "Holandés Errante", que se cree que es un espejismo superior) son consecuencias directas de la mezcla turbulenta y de los gradientes de índice de refracción resultantes que afectan a la luz visible e infrarroja (\cite{young_mirages_nodate}).

Al pasar a longitudes de onda más largas, las turbulencias afectan considerablemente a las radiocomunicaciones. Aunque menos espectaculares que los efectos ópticos, las turbulencias atmosféricas pueden provocar centelleo (rápidas fluctuaciones de la amplitud y la fase de la señal), desvanecimiento y desviación del haz en las transmisiones por microondas y ondas milimétricas. Estos efectos son especialmente pronunciados en los sistemas de comunicación óptica en espacio libre en los que los haces láser se propagan a través de la atmósfera. La variación del índice de refracción a lo largo de la trayectoria de propagación, provocada por remolinos turbulentos, puede degradar considerablemente la calidad y fiabilidad de la señal, lo que exige técnicas avanzadas de modulación y compensación.

Más allá de los entornos atmosférico y terrestre, el ámbito submarino ofrece otro ejemplo convincente con el sonar (Sound Navigation and Ranging). Aunque el sonar utiliza ondas acústicas, los principios físicos subyacentes que rigen la propagación a través de un medio turbulento comparten fuertes paralelismos con el comportamiento de las ondas EM. Las primeras operaciones de sonar naval durante la Primera y la Segunda Guerra Mundial se enfrentaron con frecuencia a variaciones impredecibles en el alcance de detección y la claridad, que más tarde se atribuyeron, en parte, a la turbulencia oceánica (\cite{kajiwara_maritime_2024}). Estos remolinos turbulentos, derivados de factores como la temperatura, la salinidad y las variaciones de las corrientes, introducen inhomogeneidades en el perfil de velocidad del fluido.

\subsection{Estado del arte}

En cuanto a métodos numéricos para simular la propagación de ondas electromagnéticas, el método de diferencias finitas en el dominio del tiempo (FDTD por sus siglas en inglés) es un método central en el campo debido a su gran versatilidad y potencia. El texto de referencia de este método es \cite{taflove_computational_2010}. Es posible aplicar este método al problema de propagación de ondas en turbulencia (tanto electromagnéticas como de presión) en diversas condiciones físicas. Se expondrán varios artículos donde se ha realizado precisamente esto. 

En el artículo \cite{moss_finite-difference_2002} se introduce un esquema FDTD tridimensional para modelar la dispersión de objetos incrustados en medios aleatorios continuos, como los suelos no homogéneos. Este trabajo, que utiliza el análisis de Monte Carlo, obtiene resultados sobre la atenuación de las ondas, la dispersión y las fluctuaciones de fase en entornos naturales como la atmósfera o los medios geofísicos, y pretende ayudar a interpretar los datos de campo dispersos y a diseñar algoritmos de supresión de interferencias. Para generar el medio aleatorio, se definen una pareja de transformadas de Fourier 3D, mostrando una excelente concordancia con los resultados teóricos (la aproximación de Born). También investiga el promediado de frecuencias como técnica para discriminar objetos enterrados reduciendo las interferencias del medio aleatorio.

En la disertación \cite{williams_full-wave_2014} se presenta EMIT-3D, un nuevo código FDTD tridimensional para simulaciones de la propagación electromagnética en plasmas magnetizados, especialmente relevante para el calentamiento por microondas y las aplicaciones de diagnóstico en tokamaks. La investigación estudia cómo afectan a la propagación los filamentos de densidad (“blobs”) y la turbulencia del plasma, observando una desviación significativa del haz incluso desde filamentos de densidad poco críticos y descubriendo que la dispersión alcanza su punto máximo cuando el tamaño de los remolinos turbulentos se aproxima a la longitud de onda del haz. Este trabajo contribuye a comprender el impacto de las inhomogeneidades del plasma en los haces de microondas, que pueden afectar al rendimiento del diagnóstico y a la eficiencia del calentamiento. El código EMIT-3D incorpora perfiles realistas de turbulencia en el borde del tokamak generados por el código BOUT++ e incluye desarrollos numéricos y un nuevo análisis de estabilidad para el algoritmo FDTD.

En el artículo \cite{kinefuchi_prediction_2013} se analiza la simulación de la transmisión de microondas a través de penachos ionizados de cohetes utilizando un enfoque combinado dinámica de fluidos computacional y FDTD para estimar la atenuación de la señal. Atribuye la interferencia a los electrones libres de alta densidad en el escape y refina el modelo CFD para simulaciones en vuelo a fin de incluir la combustión turbulenta no equilibrada. El modelo FDTD dependiente de la frecuencia se utiliza para actualizar el campo eléctrico en la región del plasma, revelando que las ondas de radio interactúan fuertemente con el núcleo de alta densidad del penacho y se reflejan en él, lo que da lugar a un aspecto trapezoidal desde la perspectiva de la radiofrecuencia. La fuente también señala que la teoría de difracción simple puede ofrecer estimaciones aproximadas de atenuación para el diseño preliminar de la trayectoria de vuelo, requiriendo menos recursos computacionales.

En el informe \cite{wilson_finite-difference_2004} se explora las técnicas FDTD para simular la propagación del sonido en atmósferas dinámicas, abordando específicamente el reto de incorporar un medio de fondo en movimiento como el viento y la turbulencia. Destaca que el método FDTD convencional de rejilla escalonada no es adecuado para medios en movimiento debido a los términos advectivos, por lo que se necesitan operadores numéricos alternativos como Runge-Kutta. El estudio llega a la conclusión de que las soluciones rigurosas en un medio en movimiento requieren al menos el doble de memoria informática que el método FDTD estándar, y ofrece comparaciones empíricas de distintos esquemas de integración temporal, identificando algunos como más precisos y estables que otros.


Esta selección de artículos da lugar al presente proyecto, que pretende realizar una exploración y comparación del método numérico FDTD, junto con una revisión bibliográfica de métodos analíticos.