\documentclass[a4paper,12pt]{article}
\usepackage[spanish, es-tabla, english]{babel}% 'cuadro' es el título del caption de table por defecto siguiendo indicaciones de RAE; es-tabla lo cambia a 'tabla'
\usepackage[utf8]{inputenc}
\usepackage[dvipsnames,svgnames,table]{xcolor}
\usepackage{caption} % También puede ayudar con captions más complejos
\usepackage{physics}
\usepackage{amssymb}
\usepackage{amsmath}
\usepackage{mathptmx}
\usepackage{url} % Agregar el paquete url
\usepackage{blindtext}
\usepackage{fancyhdr}
\usepackage{microtype}
\usepackage{float}
\usepackage{tikz}
\usepackage{pgfplots}
\pgfplotsset{compat=1.18}
\usepackage{xcolor, graphicx}
\usepackage{tabularx} % https://www.overleaf.com/learn/latex/Tables#Tables_with_a_fixed_width
\usepackage{tgbonum}
\usepackage{multicol}
\usepackage{lmodern}
\usepackage{makecell} % Agregar en el preámbulo
\usepackage{tcolorbox}
\usepackage{colortbl}
\usepackage[T1]{fontenc}
\usepackage[lmargin=3cm,rmargin=3cm]{geometry}
\usepackage[figurename=Fig., labelfont=it]{caption}
\usepackage{subcaption}
\usepackage{lipsum}
% https://tex.stackexchange.com/questions/50747/options-for-appearance-of-links-in-hyperref
\usepackage[hidelinks]{hyperref}
% por si quieres cambiar configurar el color de los enlaces
    %\hypersetup{
    %colorlinks=true,
    %linkcolor=RoyalBlue,
    %filecolor=magenta,
    %urlcolor=gray,
    %citecolor=gray,
    %pdfpagemode=FullScreen
    %}
\usepackage{parskip} % permite usar Enter como \\
\usepackage{listings} % Para incluir codiog, https://www.overleaf.com/learn/latex/Code_listing#Using_listings_to_highlight_code
\urlstyle{same}

% ------------------------------------------ Bibliografía ----------------------------------------

% https://www.overleaf.com/learn/latex/Bibliography_management_in_LaTeX
\usepackage{csquotes}
\usepackage[
bibencoding=auto,
backend=biber,
style=numeric,
sorting=ynt,
]{biblatex} % Imports biblatex package. Configura el encoding, el backend, da el estilo de citas y selecciona la forma de ordenar (year, name, title)
% posibles estilos: https://www.overleaf.com/learn/latex/Biblatex_bibliography_styles
\addbibresource{TFG.bib} % Import the bibliography file

% -------------------------------------------- Sangria -------------------------------------------

\setlength{\parskip}{6pt}
\setlength{\parindent}{0pt}

% -------------------------------------------- Figuras -------------------------------------------

\makeatletter
\long\def\@makecaption#1#2{%
  \vskip\abovecaptionskip
  \sbox\@tempboxa{#1. #2}%
  \ifdim \wd\@tempboxa >\hsize
    #1. #2\par
  \else
    \global \@minipagefalse
    \hb@xt@\hsize{\hfil\box\@tempboxa\hfil}%
  \fi
  \vskip\belowcaptionskip}
\makeatother

% ------------------------------------------- Geometría ------------------------------------------

\geometry{top=2cm, bottom=2cm, left=2cm, right=2cm}

% ----------------------------- Números de las ecuaciones y figuras --------------------------------
% Configura para que aparezcan en todas las ecuaciones la sección, la subsección y el número de la ecuación
\numberwithin{equation}{subsection}
\numberwithin{figure}{subsection}

% ------------------------------------------- Códigos ----------------------------------------------

% Define un estilo más bonito para el codigo
\definecolor{codegreen}{rgb}{0,0.6,0}
\definecolor{codegray}{rgb}{0.5,0.5,0.5}
\definecolor{codepurple}{rgb}{0.58,0,0.82}
\definecolor{backcolour}{rgb}{0.95,0.95,0.92}

\lstdefinestyle{mystyle}{
    backgroundcolor=\color{backcolour},   
    commentstyle=\color{codegreen},
    keywordstyle=\color{magenta},
    numberstyle=\tiny\color{codegray},
    stringstyle=\color{codepurple},
    basicstyle=\ttfamily\footnotesize,
    breakatwhitespace=false,         
    breaklines=true,                 
    captionpos=b,                    
    keepspaces=true,                 
    numbers=left,                    
    numbersep=5pt,                  
    showspaces=false,                
    showstringspaces=false,
    showtabs=false,                  
    tabsize=2
}
% Para que funcione bien el utf8 encoding
% https://es.wikibooks.org/wiki/Manual_de_LaTeX/Listados_de_c%C3%B3digo/Listados_con_listings#Problemas_de_codificaci%C3%B3n_(%C2%A1%C2%A1%C2%A1IMPORTANTE_EN_ESPA%C3%91OL!!!)
\lstset{literate=
  {á}{{\'a}}1 {é}{{\'e}}1 {í}{{\'i}}1 {ó}{{\'o}}1 {ú}{{\'u}}1
  {Á}{{\'A}}1 {É}{{\'E}}1 {Í}{{\'I}}1 {Ó}{{\'O}}1 {Ú}{{\'U}}1
  {à}{{\`a}}1 {è}{{\`e}}1 {ì}{{\`i}}1 {ò}{{\`o}}1 {ù}{{\`u}}1
  {À}{{\`A}}1 {È}{{\'E}}1 {Ì}{{\`I}}1 {Ò}{{\`O}}1 {Ù}{{\`U}}1
  {ä}{{\"a}}1 {ë}{{\"e}}1 {ï}{{\"i}}1 {ö}{{\"o}}1 {ü}{{\"u}}1
  {Ä}{{\"A}}1 {Ë}{{\"E}}1 {Ï}{{\"I}}1 {Ö}{{\"O}}1 {Ü}{{\"U}}1
  {â}{{\^a}}1 {ê}{{\^e}}1 {î}{{\^i}}1 {ô}{{\^o}}1 {û}{{\^u}}1
  {Â}{{\^A}}1 {Ê}{{\^E}}1 {Î}{{\^I}}1 {Ô}{{\^O}}1 {Û}{{\^U}}1
  {ã}{{\~a}}1 {ẽ}{{\~e}}1 {ĩ}{{\~i}}1 {õ}{{\~o}}1 {ũ}{{\~u}}1
  {Ã}{{\~A}}1 {Ẽ}{{\~E}}1 {Ĩ}{{\~I}}1 {Õ}{{\~O}}1 {Ũ}{{\~U}}1
  {œ}{{\oe}}1 {Œ}{{\OE}}1 {æ}{{\ae}}1 {Æ}{{\AE}}1 {ß}{{\ss}}1
  {ű}{{\H{u}}}1 {Ű}{{\H{U}}}1 {ő}{{\H{o}}}1 {Ő}{{\H{O}}}1
  {ç}{{\c c}}1 {Ç}{{\c C}}1 {ø}{{\o}}1 {å}{{\r a}}1 {Å}{{\r A}}1
  {€}{{\euro}}1 {£}{{\pounds}}1 {«}{{\guillemotleft}}1
  {»}{{\guillemotright}}1 {ñ}{{\~n}}1 {Ñ}{{\~N}}1 {¿}{{?`}}1 {¡}{{!`}}1 
}

\begin{document}

% Preambulo del TFG en si mismo: portada, titulo, configuración del encabezado, página resumen, abstract en español e inglés, e indices de tablas y figuras

\begin{titlepage}
% --------------------------------------------- Título -------------------------------------------

    \includegraphics[width=\linewidth]{figures/UEM-Logo-letra2.png}

    % Espacio vertical
    \vspace*{1cm}
    
    \begin{center}
        \Large\textbf{UNIVERSIDAD EUROPEA DE MADRID} 
        
        \vspace{2cm}
        \large\textbf{ESCUELA DE ARQUITECTURA, INGENIERÍA Y DISEÑO} 
        
        \vspace{1.3cm}
        \large\textbf{GRADO EN FÍSICA} 
        
        \vspace{3cm}
        \large TRABAJO FIN DE GRADO 
        
        \vspace{1.4cm}
        \Large\textbf{Efecto de la turbulencia en propagación de IR}
        
    \end{center}
    
% --------------------------------------------- Autores ------------------------------------------
    \begin{center}
        \vspace{1.2cm}
        \large JORGE SAN JOSÉ VILLAR 
        
        Dirigido por
        
        Dr. RODRIGO BLASCO CHICANO
    \end{center}
    
% ---------------------------------------------- Fecha -------------------------------------------

    \begin{center}
        CURSO 2024-2025
    \end{center}

\end{titlepage}

%\rule{\textwidth}{0.8pt} % rompe la segunda página

% ------------------------------------- Configuración de encabezado ---------------------------------

\setlength{\headheight}{30.14165pt} % Ajusta la altura del encabezado
%\addtolength{\topmargin}{-18.14165pt} % Ajusta el margen superior (opcional)

\pagestyle{fancy} % Activa el uso de fancyhdr
\fancyhf{} % Borra encabezados y pies de página predeterminados

% Encabezado izquierdo con texto
\fancyhead[L]{\textbf{Efecto de la turbulencia en propagación de IR} \\ Jorge San José}

% Encabezado derecho con imagen (ajusta la altura según necesites)
\fancyhead[R]{\includegraphics[width=2.5cm]{figures/UEM-Logo-Letra.png}}

% Pie de página con número de página centrado
\fancyfoot[C]{\thepage}

% Evitar encabezados en páginas en blanco
\makeatletter
\renewcommand{\cleardoublepage}{\clearpage\thispagestyle{plain}}
\makeatother

\setlength{\columnsep}{1cm} % o ajusta a tu gusto
\setlength{\columnseprule}{0pt} % Línea entre columnas, si quieres una fina usa 0.2pt

\raggedcolumns

% ------------------------------------- Pagina resumen titulo ---------------------------------------

TÍTULO: EFECTO DE LA TURBULENCIA EN PROPAGACIÓN DE IR 

\vspace*{1cm}
AUTOR: JORGE SAN JOSÉ VILLAR

\vspace*{1cm}
TITULACIÓN: GRADO EN FÍSICA 

\vspace*{1cm}
DIRECTOR DEL PROYECTO: Dr. RODRIGO BLASCO CHICANO

\vspace*{1cm}
FECHA: JUNIO de 2025

\newpage

% ----------------------------------- Abstract: español e inglés ------------------------------------

\addcontentsline{toc}{section}{Resumen}
\section*{Resumen}

\begin{otherlanguage}{spanish}
    \begin{abstract}
        Este proyecto final de grado aborda el problema de la propagación de ondas electromagnéticas, específicamente infrarrojas (IR), a través de medios turbulentos, como la atmósfera. Reconociendo la importancia de comprender estos fenómenos para optimizar sistemas en entornos dinámicos y heterogéneos, el estudio combina metodologías teóricas y numéricas. Teóricamente, se exploran el enfoque de la óptica geométrica para ondas cortas y métodos perturbativos de la ecuación de onda para distancias mayores, revelando cómo las fluctuaciones del índice de refracción afectan la amplitud y fase de las ondas. Numéricamente, se implementa el algoritmo de Diferencias Finitas en el Dominio del Tiempo (FDTD), utilizando Python y la librería FDTD de Floris Laporte, para simular la propagación en un medio con permitividad aleatoria que modela la turbulencia del aire. Los principales resultados incluyen el análisis de la variación cuadrática media del logaritmo de la amplitud en función de la escala de la turbulencia, con simulaciones que corroboran las predicciones teóricas. La principal conclusión es la validación de un enfoque simple que permite modelar y predecir los efectos de la turbulencia en la propagación de IR, sentando las bases para futuras aplicaciones en el análisis de señales e imágenes.
    \end{abstract}
\end{otherlanguage}

\textbf{Palabras clave}: \textbf{Turbulencia, Ondas Electromagnéticas, Dinámica de Fluidos, Métodos Perturbativos, Método FDTD}

\newpage

\addcontentsline{toc}{section}{Abstract}
\section*{Abstract}

\begin{otherlanguage}{english}
    \begin{abstract}
        This final degree project addresses the problem of electromagnetic wave propagation, specifically infrared (IR), through turbulent media, such as the atmosphere. Recognizing the importance of understanding these phenomena to optimize systems in dynamic and heterogeneous environments, the study combines theoretical and numerical methodologies. Theoretically, the geometrical optics approach for short waves and perturbative wave equation methods for longer distances are explored, revealing how refractive index fluctuations affect wave amplitude and phase. Numerically, the Finite Difference Time Domain (FDTD) algorithm is implemented, using Python and Floris Laporte's FDTD library, to simulate propagation in a medium with random permittivity that models air turbulence. The main results include the analysis of the root mean square variation of the logarithm of the amplitude as a function of the scale of the turbulence, with simulations that corroborate the theoretical predictions. The main conclusion is the validation of a simple approach to model and predict the effects of turbulence on IR propagation, laying the foundation for future applications in signal and image analysis.
    \end{abstract}
    \textbf{Key words}: \textbf{Turbulence, Electromagnetic Waves Fluid Dynamics, Perturbation Methods, FDTD Method}
\end{otherlanguage}

\newpage

% ------------------------------------------ Tabla resumen ------------------------------------------

\addcontentsline{toc}{section}{Tabla resumen}
\section*{Tabla resumen}

\begin{otherlanguage}{spanish}
\begin{table}[h]
    \centering
    \begin{tabularx}{1.0\textwidth} { | >{\centering\arraybackslash}X | >{\centering\arraybackslash}X | }
        \hline
         & \textbf{Datos} \\
        \hline
        \textbf{Nombre y apellidos:} & Jorge San José Villar \\
        \hline
        \textbf{Título del proyecto:} & Efecto de la turbulencia en propagación de IR \\
        \hline
        \textbf{Directores del proyecto:} & Dr. Rodrigo Blasco Chicano\\
        \hline
        \textbf{El proyecto se ha realizado en colaboración de una empresa o a petición de una empresa:} & NO \\
        \hline
        \textbf{El proyecto ha implementado un producto:} & NO \\
        \hline
        \textbf{El proyecto ha consistido en el desarrollo de una investigación o innovación:} & SI \\
        \hline
        \textbf{Objetivo general del proyecto:} & Investigar y comprender el efecto de la turbulencia en la propagación de ondas infrarrojas (IR) mediante métodos teóricos y numéricos.\\
        \hline
    \end{tabularx}
    \caption{Tabla resumen del proyecto}
    \label{tab:tabla_resumen}
\end{table}
\end{otherlanguage}


\newpage

% ----------------------------------------------- Índices -------------------------------------------
% indice normal
\renewcommand*\contentsname{Índice}
\tableofcontents % Si necesitas tabla de contenido
\newpage

% indice de figuras
\addcontentsline{toc}{section}{Índice de figuras}
\begin{otherlanguage}{spanish}
    \listoffigures
\end{otherlanguage}

\newpage

% indice de tablas
\addcontentsline{toc}{section}{Índice de tablas}
\begin{otherlanguage}{spanish}
    \listoftables
\end{otherlanguage}

\section{Introducción}

La propagación de ondas electromagnéticas (EM) a través de medios turbulentos plantea un reto polifacético en diversas disciplinas científicas y de ingeniería. Históricamente, las observaciones de los fenómenos inducidos por la turbulencia han abarcado desde el desenfoque macroscópico de los objetos celestes hasta la sutil degradación de las señales de comunicación. Comprender estos efectos es primordial para optimizar el rendimiento de los sistemas, sobre todo en aplicaciones en las que el medio de propagación es intrínsecamente dinámico y heterogéneo.

En el ámbito de la óptica, la manifestación más emblemática y observada de la turbulencia es la visión astronómica (\cite{tubbs_lucky_2003}). Desde la antigüedad, el parpadeo de las estrellas y la claridad variable de las imágenes telescópicas se han atribuido a irregularidades atmosféricas (\cite{noauthor_mystery_1994}). Ya en 1665, Robert Hooke especuló sobre el papel de la variación del índice de refracción de la atmósfera. Más tarde, a finales del siglo XIX y principios del XX, la turbulencia atmosférica fue célebremente implicada en la interpretación errónea de las características marcianas como «canales», poniendo de relieve su profundo impacto en la percepción visual y la observación científica. La percepción de estos «canales» en Marte por parte de astrónomos como Giovanni Schiaparelli y Percival Lowell se atribuye ahora en parte a ilusiones ópticas exacerbadas por las turbulencias atmosféricas. Del mismo modo, el fenómeno común de la neblina térmica sobre superficies calientes y los espejismos más complejos (incluido el legendario "Holandés Errante", que se cree que es un espejismo superior) son consecuencias directas de la mezcla turbulenta y de los gradientes de índice de refracción resultantes que afectan a la luz visible e infrarroja (\cite{young_mirages_nodate}).

Al pasar a longitudes de onda más largas, las turbulencias afectan considerablemente a las radiocomunicaciones. Aunque menos espectaculares que los efectos ópticos, las turbulencias atmosféricas pueden provocar centelleo (rápidas fluctuaciones de la amplitud y la fase de la señal), desvanecimiento y desviación del haz en las transmisiones por microondas y ondas milimétricas. Estos efectos son especialmente pronunciados en los sistemas de comunicación óptica en espacio libre en los que los haces láser se propagan a través de la atmósfera. La variación del índice de refracción a lo largo de la trayectoria de propagación, provocada por remolinos turbulentos, puede degradar considerablemente la calidad y fiabilidad de la señal, lo que exige técnicas avanzadas de modulación y compensación.

Más allá de los entornos atmosférico y terrestre, el ámbito submarino ofrece otro ejemplo convincente con el sonar (Sound Navigation and Ranging). Aunque el sonar utiliza ondas acústicas, los principios físicos subyacentes que rigen la propagación a través de un medio turbulento comparten fuertes paralelismos con el comportamiento de las ondas EM. Las primeras operaciones de sonar naval durante la Primera y la Segunda Guerra Mundial se enfrentaron con frecuencia a variaciones impredecibles en el alcance de detección y la claridad, que más tarde se atribuyeron, en parte, a la turbulencia oceánica (\cite{kajiwara_maritime_2024}). Estos remolinos turbulentos, derivados de factores como la temperatura, la salinidad y las variaciones de las corrientes, introducen inhomogeneidades en el perfil de velocidad del fluido.

\subsection{Estado del arte}

En cuanto a métodos numéricos para simular la propagación de ondas electromagnéticas, el método de diferencias finitas en el dominio del tiempo (FDTD por sus siglas en inglés) es un método central en el campo debido a su gran versatilidad y potencia. El texto de referencia de este método es \cite{taflove_computational_2010}. Es posible aplicar este método al problema de propagación de ondas en turbulencia (tanto electromagnéticas como de presión) en diversas condiciones físicas. Se expondrán varios artículos donde se ha realizado precisamente esto. 

En el artículo \cite{moss_finite-difference_2002} se introduce un esquema FDTD tridimensional para modelar la dispersión de objetos incrustados en medios aleatorios continuos, como los suelos no homogéneos. Este trabajo, que utiliza el análisis de Monte Carlo, obtiene resultados sobre la atenuación de las ondas, la dispersión y las fluctuaciones de fase en entornos naturales como la atmósfera o los medios geofísicos, y pretende ayudar a interpretar los datos de campo dispersos y a diseñar algoritmos de supresión de interferencias. Para generar el medio aleatorio, se definen una pareja de transformadas de Fourier 3D, mostrando una excelente concordancia con los resultados teóricos (la aproximación de Born). También investiga el promediado de frecuencias como técnica para discriminar objetos enterrados reduciendo las interferencias del medio aleatorio.

En la disertación \cite{williams_full-wave_2014} se presenta EMIT-3D, un nuevo código FDTD tridimensional para simulaciones de la propagación electromagnética en plasmas magnetizados, especialmente relevante para el calentamiento por microondas y las aplicaciones de diagnóstico en tokamaks. La investigación estudia cómo afectan a la propagación los filamentos de densidad (“blobs”) y la turbulencia del plasma, observando una desviación significativa del haz incluso desde filamentos de densidad poco críticos y descubriendo que la dispersión alcanza su punto máximo cuando el tamaño de los remolinos turbulentos se aproxima a la longitud de onda del haz. Este trabajo contribuye a comprender el impacto de las inhomogeneidades del plasma en los haces de microondas, que pueden afectar al rendimiento del diagnóstico y a la eficiencia del calentamiento. El código EMIT-3D incorpora perfiles realistas de turbulencia en el borde del tokamak generados por el código BOUT++ e incluye desarrollos numéricos y un nuevo análisis de estabilidad para el algoritmo FDTD.

En el artículo \cite{kinefuchi_prediction_2013} se analiza la simulación de la transmisión de microondas a través de penachos ionizados de cohetes utilizando un enfoque combinado dinámica de fluidos computacional y FDTD para estimar la atenuación de la señal. Atribuye la interferencia a los electrones libres de alta densidad en el escape y refina el modelo CFD para simulaciones en vuelo a fin de incluir la combustión turbulenta no equilibrada. El modelo FDTD dependiente de la frecuencia se utiliza para actualizar el campo eléctrico en la región del plasma, revelando que las ondas de radio interactúan fuertemente con el núcleo de alta densidad del penacho y se reflejan en él, lo que da lugar a un aspecto trapezoidal desde la perspectiva de la radiofrecuencia. La fuente también señala que la teoría de difracción simple puede ofrecer estimaciones aproximadas de atenuación para el diseño preliminar de la trayectoria de vuelo, requiriendo menos recursos computacionales.

En el informe \cite{wilson_finite-difference_2004} se explora las técnicas FDTD para simular la propagación del sonido en atmósferas dinámicas, abordando específicamente el reto de incorporar un medio de fondo en movimiento como el viento y la turbulencia. Destaca que el método FDTD convencional de rejilla escalonada no es adecuado para medios en movimiento debido a los términos advectivos, por lo que se necesitan operadores numéricos alternativos como Runge-Kutta. El estudio llega a la conclusión de que las soluciones rigurosas en un medio en movimiento requieren al menos el doble de memoria informática que el método FDTD estándar, y ofrece comparaciones empíricas de distintos esquemas de integración temporal, identificando algunos como más precisos y estables que otros.


Esta selección de artículos da lugar al presente proyecto, que pretende realizar una exploración y comparación del método numérico FDTD, junto con una revisión bibliográfica de métodos analíticos.

\section{Objetivos}

\subsection{Objetivos generales}

El objetivo del presente trabajo final de grado es exponer y obtener resultados en la propagación de ondas electromagnéticas en un medio turbulento, tanto teóricamente como a través del método numérico FDTD. 

\subsection{Objetivos específicos}

En cuanto a objetivos específicos podemos mencionar los siguientes:

\begin{itemize}
    \item Realizar una revisión bibliográfica y teórica tanto sobre las bases de radiación electromagnética y mecánica de fluidos como los principales problemas del campo y las formas en las que se resuelven. 
    \item Uso de herramientas matemáticas avanzadas, especialmente aquellas que no han aparecido en el temario de la carrera, como por ejemplo transformaciones de Mellin y funciones hipergeométicas.
    \item Implementación de los métodos numéricos expuestos o en su defecto una exploración del software más utilizado en el circulo académico.
    \item Contraste entre los resultados de métodos numéricos y métodos analíticos. 
    \item Utilizar los resultados obtenidos para informar análisis consiguiente, principalmente en el ámbito del análisis de la señal e imagen. 
\end{itemize}

\subsection{Beneficios del proyecto}

El proyecto implica una revisión bibliográfica y teórica de la radiación electromagnética y la mecánica de fluidos, así como la aplicación de herramientas matemáticas avanzadas principalmente métodos perturbativos y espectrales. Esto contribuye a una comprensión más profunda de los fenómenos físicos subyacentes a la propagación de ondas en medios complejos.

El uso del método FDTD (Diferencia Finita en el Dominio del Tiempo) a través de una librería de Python para simular la propagación de ondas en turbulencia y el contraste de estos resultados con métodos analíticos, valida una potente herramienta numérica para abordar este tipo de problemas. Este desarrollo es valioso dada la versatilidad y el poder del método FDTD en el campo de la simulación electromagnética.

\section{Desarrollo del trabajo de fin de grado}

\subsection{Planificación del trabajo de fin de grado}

Las tareas/actividades realizadas son:

\begin{itemize}
    \item  Revisión y expansión de la bibliografía.
    \item Estudio de la viabilidad de implementar los métodos numéricos en un lenguaje genérico ó utilizar un software ya establecido. Búsqueda de posibles implementaciones en repositorios de código públicos que puedan servir de ejemplo.
    \item Aprendizaje de la librería FDTD y desarrollo del código de simulación. El desarrollo del código involucra desarrollar la simulación y la creación de figuras para el trabajo.
    \item Redacción y composición tipográfica del proyecto.
\end{itemize}

El cronograma inicial del proyecto se puede ver en la figura \ref{fig:cronograma}. Sin embargo, las actividades relacionadas con la obtención de resultados resultaron más difíciles de lo anticipado, por lo que el horario entero debió ser retrasado. 

\begin{figure}[h]
    \centering
    \includegraphics[width=0.65\linewidth]{figures/Cronograma1.png}
    \caption{Cronograma inicial de las actividades del proyecto}
    \label{fig:cronograma}
\end{figure}

\subsection{Descripción de la solución, metodologías y herramientas empleadas}

El proyecto aborda la problemática de la propagación de ondas electromagnéticas en medios turbulentos, específicamente las ondas infrarrojas (IR), empleando un enfoque que combina métodos teóricos y numéricos.

La solución busca investigar y comprender el efecto de la turbulencia en la propagación de ondas infrarrojas (IR).Esto se logra mediante la modelización del medio turbulento y la simulación de la interacción de las ondas electromagnéticas con este medio.

En cuanto a los métodos teóricos, se ha realizado una revisión bibliográfica y teórica de las bases de la radiación electromagnética y la mecánica de fluidos, así como los principales problemas del campo y sus soluciones. Dentro de este enfoque se exploran dos aproximaciones, ambas utilizando métodos perturbativos: una aproximación basada en rayos y otra basada en la ecuación de onda.
    
En los métodos numéricos el enfoque ha sido utilización algoritmo de Diferencias Finitas en el Dominio del Tiempo (FDTD). Esta metodología implica discretización, actualización de campos, condiciones de contorno y modelado del medio.

Las herramientas de sofware utilizadas son:

\begin{itemize}
    \item \LaTeX: redacción y composición tipográfica del documento.
    \item Python: Lenguaje de programación utilizado para la mayor parte de la simulación numérica
    \item Floris Laporte's FDTD Library: Una librería de Python específica para simulaciones FDTD
    \item SciPy: Librería de Python para computación científica, utilizada para remuestreo, interpolación y convoluciones.
    \item Numpy: Librería de Python para cálculos numéricos, utilizada para la manipulación de arrays y operaciones matemáticas.
    \item Git y Github: Control de versiones y alojamiento de archivos.
\end{itemize}

Se utilizó "12 steps to Navier-Stokes" \cite{barba_cfd_2018} así como "Lectures in elementary fluid dynamics: physics, mathematics and applications" \cite{mcdonough_lectures_2009} como recursos didácticos. Se consultó también la base de datos "Refractiveindex.info" para el modelaje del indice de refracción.

\section{Fundamentos teóricos}

\subsection{Funciones de Green}\label{subsect:green}

Las funciones de Green son una herramienta matemática utilizada para resolver ecuaciones diferenciales lineales no homogéneas (\cite{morse_methods_1999}, \cite{farlow_partial_1993}). Una función de Green, denotada como $G(\vec{r}, \vec{r}')$, es la "respuesta al impulso" de un operador diferencial lineal. Esto significa que es la solución a la ecuación diferencial cuando el «término forzante» o «fuente» es una función delta de Dirac, $\delta(\vec{r} - \vec{r}')$, que representa una fuente puntual (una entrada infinitamente aguda y localizada).

Para un operador diferencial lineal $\mathcal{L}$, la función de Green satisface:

\begin{equation}\label{ec:green_diferencial}
    \mathcal{L} G(\vec{r}, \vec{r}') = \delta(\vec{r} - \vec{r}')
\end{equation}

donde $\vec{r}$ es el punto de observación y $\vec{r}'$ es el punto de origen.

Una vez que se tiene la función de Green para un operador dado y las condiciones de contorno, se puede encontrar la solución $u(\vec{r})$ a cualquier ecuación no homogénea:

\begin{equation}
    \mathcal{L} u(\vec{r}) = f(\vec{r})
\end{equation}

integrando la función de Green multiplicada por el término fuente $f(\vec{r}')$ en todo el dominio:

\begin{equation}
    u(\vec{r}) = \int G(\vec{r}, \vec{r}') f(\vec{r}') dV'
\end{equation}

Esta integral suma esencialmente las respuestas de todas las fuentes puntuales infinitesimales que componen la fuente distribuida $f(\vec{r}')$. Esto se basa en la linealidad del operador diferencial, lo que permite la superposición de soluciones.

La solución de la Ecuación de Helmholtz inhomogénea, que viene dada por:

\begin{equation}
    \nabla^2u + k^2 u = f(\vec{r})
\end{equation}

puede expresarse en términos de una integral utilizando una función de Green, que utilizando \ref{ec:green_diferencial} debe seguir:

\begin{equation}
    \nabla^2 G(\vec{r}, \vec{r}') + k^2 G(\vec{r}, \vec{r}') = -\delta(\vec{r} - \vec{r}')
\end{equation}

La convención del signo para la función delta en el lado derecho puede variar en la literatura, a menudo siendo $-4\pi\delta(\vec{r} - \vec{r}')$ en tres dimensiones por coherencia con la electrostática, pero el principio sigue siendo el mismo.

La forma exacta de la función de Green depende de la dimensión del problema y las condiciones de contorno. Para el caso tridimensional se tiene:

\begin{equation}
    G(\vec{r}, \vec{r}') = \frac{e^{ik|\vec{r} - \vec{r}'|}}{4\pi|\vec{r} - \vec{r}'|}
\end{equation}

Otras convenciones en cuanto al signo de la exponencial son posibles. En cambio, para una dimensión espacial se tiene:

\begin{equation}
    G(x, x') = \frac{e^{ik|x - x'|}}{2ik}
\end{equation}

Lo que pone de manifiesto la relación de esta ecuación con la resolución de la ecuación de ondas en el espacio de frecuencias, siendo $e^{ik|x - x'|}$ la función base o autofunción con la que se puede representar cualquier otra función. Esto se hace comúnmente al transformar al espacio de momentos en la física cuántica.

\subsection{Electromagnetismo}

Como ya se ha visto, en el vacío las Ecuaciones de Maxwell toman la forma (\cite{griffiths_introduction_2017}, \cite{jackson_classical_1975}, \cite{landau_electrodynamics_1993}): 

\begin{equation}\label{ec:maxwell_vacio_div}
    \nabla\cdot \mathbf{E} = 0 \quad \nabla\cdot \mathbf{B} = 0 
\end{equation}

\begin{equation}\label{ec:maxwell_vacio_rot}
    \nabla\times\mathbf{E} = - \frac{\partial\mathbf{B}}{\partial t}; \quad \nabla\times \mathbf{B} = \mu_0\epsilon_0 \frac{\partial\mathbf{E}}{\partial t}
\end{equation}

Realizando un promedio de las propiedades microscópicas de un material, es posible llegar a las llamadas ecuaciones de Maxwell en la materia. Se introducen los campos auxiliares de desplazamiento eléctrico $\mathbf{D}$ y excitación magnética $\mathbf{H}$, con lo que queda:

\begin{equation}\label{ec:maxwell_materia_div}
    \nabla\cdot \mathbf{D} = \rho_f \quad \nabla\cdot \mathbf{B} = 0 
\end{equation}

\begin{equation}\label{ec:maxwell_materia_rot}
    \nabla\times\mathbf{E} = - \frac{\partial\mathbf{B}}{\partial t}; \quad \nabla\times \mathbf{H} = \mathbf{J}_f+ \frac{\partial\mathbf{D}}{\partial t}
\end{equation}

\subsection{Métodos espectrales en la propagación de ondas}\label{subsec:espectral_onda}

Una de las posibles soluciones a las ecuaciones de Maxwell \ref{ec:maxwell_vacio_div}, \ref{ec:maxwell_vacio_rot} toman la forma de ondas, lo que se puede ver fácilmente desacoplando las ecuaciones utilizando la identidad para un rotacional de un rotacional:

\begin{equation}
    \nabla\times(\nabla\times\mathbf{A}) = \nabla(\nabla\cdot\mathbf{A}) - \nabla^2\mathbf{A}
\end{equation}

Tomando el rotacional de ambas ecuaciones en \ref{ec:maxwell_vacio_rot} y utilizando tanto la identidad antes mencionada y \ref{ec:maxwell_vacio_div} se puede llegar a: 

\begin{equation}
    \mu_0\epsilon_0 \frac{\partial^2\mathbf{E}}{\partial t^2} - \nabla^2 \mathbf{E} = 0; \quad
    \mu_0\epsilon_0 \frac{\partial^2\mathbf{B}}{\partial t^2} - \nabla^2 \mathbf{B} = 0
\end{equation}

De aquí es posible reconocer la constante $\mu_0\epsilon_0$ como $\frac{1}{c^2}$, donde $c$ es la velocidad de la luz en el vacío. De esta manera, las Ecuaciones de Maxwell se han reducido a dos ecuaciones de ondas involucrando un campo vectorial. Consideremos una onda plana que se encuentra con una abertura u obstáculo de forma cualquiera. Al propagarse a través de esta, suponiendo que la longitud de onda sea de un orden no muy alejado al tamaño de las características más pequeñas de la abertura, se producirá el fenómeno de difracción, "doblando" la onda alrededor de la abertura. Para trabajar con un caso simplificado trabajaremos con la amplitud compleja de la onda $U$ relacionada con la intensidad de los campos. De esta forma tendremos: 

\begin{equation}
    \left(\nabla^2 - \frac{1}{c^2} \frac{\partial^2 }{\partial t^2}\right) U(x, y, z, t) = 0
\end{equation}

Suponiendo que tenemos luz monocromática de frecuencia angular $w = 2\pi f$ se pueden buscar soluciones de la siguiente forma: $U(x, y, z, t) = u(x, y, z)e^{-iwt}$. Esto lleva a una Ecuación de Helmholtz:

\begin{equation}
    \nabla^2 u + k^2u = 0; \quad k = \frac w c = \frac {2\pi}{\lambda}
\end{equation}

Para soluciones soluciones separables: $u(x, y, z) = f_x(x)\times f_y(y)\times f_z(z)$, se puede llegar a tres ecuaciones para cada una de las funciones $f$:

$$
    \frac{d^2}{dx^2} f_x(x) + k_x^2f_x(x) = 0;  \quad
    \frac{d^2}{dy^2} f_y(y) + k_y^2f_y(y) = 0;  \quad
    \frac{d^2}{dz^2} f_z(z) + k_z^2f_z(z) = 0  
$$

Junto con la condición: $k^2 = k_x^2 + k_y^2 + k_z^2$. Las soluciones de estas ecuaciones serán senos y cosenos o, de forma equivalente, una exponencial compleja. Volviendo a la función $u$, que pasaremos a denotar $u_s$, al ser una de las posibles soluciones separables, tenemos:

\begin{equation}
    u_s(x, y, z) = Ae^{ik_x x}e^{ik_y y}e^{ik_z z} = A e^{i(k_x x + k_y y )} e^{\pm z \sqrt{k^2 - k_x^2 - k_y^2}}
\end{equation}

Se ha resuelto $k_z$ para liberarnos de una de las $k_i$; con $A$ compleja en general. Sabemos que esta solución $u_s$ es solo una de las posibles, para encontrar la solución más general debemos construir una combinación lineal de todas las posibles soluciones con diferentes $k_x$ y $k_y$. Es posible hacer esto con una integral:

\begin{equation}
    u(x, y, z) = \int_{-\infty}^\infty \int_{-\infty}^\infty dk_x dk_y u_s(x, y, z, k_x, k_y)
\end{equation}

Se puede desarrollar esta ecuación teniendo en cuenta que la constante $A$ dependerá de $k_x$ y $k_y$:

\begin{equation}
    u(x, y, z) = \int_{-\infty}^\infty \int_{-\infty}^\infty dk_x dk_y A(k_x, k_y) e^{i(k_x x + k_y y )} e^{\pm z \sqrt{k^2 - k_x^2 - k_y^2}}
\end{equation}

Entonces en $z = 0$ se tiene:

\begin{equation}
    u(x, y, 0) = \int_{-\infty}^\infty \int_{-\infty}^\infty dk_x dk_y A(k_x, k_y) e^{i(k_x x + k_y y )}
\end{equation}

Es posible entonces reconocer que $u$ y $A$ están relacionados por una transformada de Fourier bidimensional, que escribiremos:

\begin{equation}\label{ec:coefs_prop}
    A(k_x, k_y) = \mathcal{F}[u(x, y, 0)]
\end{equation}

$A$ es la transformada de Fourier de $u(x, y, z)$ en $z = 0$. De forma más general:

\begin{equation}\label{ec:solucion_wave_Fourier_inv}
     u(x, y, z) = \mathcal{F}^{-1}[A(k_x, k_y)e^{- z \sqrt{k^2 - k_x^2 - k_y^2}}]
\end{equation}

Donde hemos escogido el signo menos, lo que tiene como significado que las ondas se propagan en dirección $z$ positivo. Utilizando las dos expresiones \ref{ec:coefs_prop} y \ref{ec:solucion_wave_Fourier_inv} podremos resolver el problema de propagación de estas ondas escalares suponiendo que sabemos la longitud de onda (o el vector de ondas $\mathbf{k}$), la velocidad de la onda y $u(x, y, z)$ en $z = 0$. El problema entonces queda reducido a calcular dos transformadas de Fourier, lo que se puede realizar con sencillez numéricamente. Un ejemplo del uso de este método para resolver la apertura circular puede verse en la figura \ref{fig:ejemplo_diff}. 

\begin{figure}
    \centering
    \includegraphics[width=0.7\linewidth]{figures/ejemplo_diff.png}
    \caption{Utilizando los métodos expuestos en \ref{subsec:espectral_onda}, es posible resolver la propagación a través de una apertura circular}
    \label{fig:ejemplo_diff}
\end{figure}

Analíticamente, se puede continuar mencionando la Ecuación de difracción de Fraunhofer. La Ecuación de difracción de Fraunhofer caracteriza el patrón de difracción de campo lejano de una onda electromagnética tras chocar con una abertura u obstáculo, y se expresa a menudo mediante una transformada de Fourier que relaciona la función de abertura con el patrón observado \cite{lipson_optical_2010}. Existe una similitud fundamental entre la difracción de Fraunhofer y la dispersión de ondas electromagnéticas, basada en la naturaleza ondulatoria de la luz. En ambos casos, una onda electromagnética interactúa con un objeto, ya sea una abertura bien definida o un obstáculo en la difracción, o una partícula o conjunto de partículas en la dispersión. En consecuencia, la onda incidente se redirige en varias direcciones. En la difracción, esta redirección se debe a la superposición e interferencia de las ondículas de Huygens procedentes de las partes no obstruidas del frente de onda. A la inversa, la dispersión implica la interacción de la onda con centros de dispersión individuales dentro del objeto, haciendo que emitan ondas secundarias. Ambos fenómenos se manifiestan como variaciones espaciales en la intensidad de la onda. Los patrones de difracción muestran máximos y mínimos distintivos resultantes de la interferencia constructiva y destructiva, mientras que los patrones de dispersión están influidos por el tamaño, la forma, la composición del dispersor y la longitud de onda de la onda incidente.

Para una apertura circular, el patrón obtenido recibe el nombre de disco de Airy, cuyo tamaño puede ser obtenido a partir de los primeros ceros del patrón (los cuales involucran la función de Bessel de primer orden de orden uno) y se puede expresar como $\sin \theta = 1.22 \frac{\lambda}{d}$.

En última instancia, la difracción puede considerarse un caso específico de dispersión en el que el objeto que provoca la redirección de la onda posee una estructura regular, como una abertura o una rejilla, lo que da lugar a una interferencia coherente y a la formación de patrones bien definidos. La dispersión es un término más amplio que engloba la interacción de las ondas electromagnéticas con diversos objetos, que pueden presentar o no tales estructuras regulares o reemisión coherente.

Partiendo del concepto general de dispersión, el caso específico de la dispersión por turbulencia implica la interacción de las ondas electromagnéticas con fluctuaciones aleatorias del índice de refracción de un medio, como la atmósfera o un fluido, provocadas por un flujo turbulento. Estas fluctuaciones, caracterizadas por remolinos de tamaños variables, actúan como multitud de centros de dispersión débil. A diferencia de la difracción por una abertura bien definida, la dispersión por turbulencia es intrínsecamente aleatoria y dinámica. La onda electromagnética incidente es dispersada en numerosas direcciones por estas variaciones del índice de refracción en constante cambio. Esto da lugar a fenómenos como el parpadeo de las estrellas debido a la dispersión de la luz estelar por la turbulencia atmosférica, o la borrosidad y distorsión de las imágenes vistas a través de medios turbulentos. 

Las propiedades estadísticas de la turbulencia, como el tamaño y la intensidad de las fluctuaciones del índice de refracción, determinan las características de la onda dispersa, incluidas sus fluctuaciones de intensidad (centelleo) y el grado de coherencia espacial. Mientras que la difracción de Fraunhofer considera la dispersión por un objeto fijo y estructurado en el campo lejano, la dispersión por turbulencia implica un medio de dispersión variable en el tiempo, descrito estadísticamente y distribuido por un volumen.

\subsection{Dinámica de fluidos}

La dinámica de fluidos es el estudio del movimiento de líquidos y gases, tratándolos como medios continuos. Aunque existen flujos suaves o laminares, prácticamente todos los flujos de fluidos de interés para científicos e ingenieros son turbulentos. La turbulencia, caracterizada por variaciones caóticas de la velocidad y la presencia de remolinos, es la regla más que la excepción en la dinámica de fluidos. Comprender la turbulencia es crucial para diversas aplicaciones, desde la mejora del diseño de vehículos y motores hasta la comprensión de los flujos fisiológicos. Sin embargo, describir matemáticamente la turbulencia sigue siendo uno de los problemas sin resolver más difíciles de la física clásica. La transición de flujo laminar a turbulento suele producirse cuando la velocidad de un flujo, el tamaño del objeto o la viscosidad del fluido alcanzan un punto crítico, que puede caracterizarse por el número de Reynolds. (\cite{mcdonough_lectures_2009}, \cite{symon_capitulo_1980})

La ecuación de Navier-Stokes, junto con la ecuación de continuidad (conservación de la masa), describe el movimiento de fluidos con viscosidad. En un campo gravitatorio constante $g$ es posible expresarla como:

\begin{equation}
    \rho \frac{d\vec{v}}{dt} = -\nabla P + \rho\vec{g} + \mu \nabla^2 \vec{v}
\end{equation}

donde $\vec{v}$ es el campo de velocidad del fluido, $\mu$ es el coeficiente de viscosidad dinámica, $P$ es la presión del fluido y $\rho$ su densidad. Se trata de una ecuación diferencial no lineal, lo que se puede ver expandiendo el término del lado izquierdo de la ecuación como la llamada derivada material: 

\begin{equation}
    \frac{d \vec{v}}{dt} = \frac{\partial \vec{v}}{\partial t} + \vec{v}\cdot (\nabla \vec v)
\end{equation}

Se define de esta manera en consistencia con la derivada total de un campo escalar, como $P$, que depende de tanto la posición como el tiempo. El término $\vec{v}\cdot (\nabla \vec v)$ es un término no lineal y es precisamente el que puede dar lugar a turbulencia. La fuerza relativa de la viscosidad puede caracterizarse con el ratio del término inercial $\rho |\frac{d\vec v}{d t}|$ y el término de viscosidad $\mu | \nabla ^2 \vec v|$:

\begin{equation}
    \frac{\textbf{inertia}}{\textbf{viscosidad}} = \frac{\rho |\frac{d\vec v}{d t}|}{\mu | \nabla ^2 \vec v|} \backsim \frac{\rho \frac{u}{T}}{\mu \frac{u}{L^2}} \backsim \frac{\rho \frac{u}{L/u}}{\mu \frac{u}{L^2}} = \frac{\rho u L}{\mu}
\end{equation}

Aquí $u$ es una velocidad característica,$L$ es una escala de longitud característica y $T = L/u$ es una escala de tiempo característica. El número de Reynolds se define como este ratio:

\begin{equation}
    R_e = \frac{\rho u L}{\mu}
\end{equation}

Cuando el número de Reynolds es bajo, la viscosidad es dominante y el flujo es laminar. En cambio, las turbulencias suelen producirse cuando el número de Reynolds es alto.

En ciertas situaciones es útil considerar el fluido como un campo aleatorio (\cite{chernov_wave_1967}, \cite{tatarski_wave_1967}). Para trabajar con estos campos debido a su gran complejidad es útil definir funciones que proporcionen ciertas características del campo. Una de las más importantes y más utilizadas es el valor medio. Otra de las más utilizadas es la función de correlación. La función de correlación caracteriza la relación estadística mutua entre las fluctuaciones de una magnitud en diferentes puntos del espacio o diferentes instantes de tiempo, dependiendo del contexto. En \cite{chernov_wave_1967}, para una cantidad escalar $\mu (x, y, z, t)$, se puede definir una función de correlación de la siguiente manera si se quiere tomar el proceso como estático en el tiempo:

\begin{equation}\label{ec:correlation_1}
    N_{12} = \overline{\mu(x_1, y_1, z_1, t)\mu(x_2, y_2, z_2, t)}
\end{equation}

Donde se debe entender la sobrebarra como el promedio temporal $\overline{f(t)} = \lim_{T\rightarrow{\infty}} \frac{1}{2T}\int_{-T}^T f(t)dt$. En \cite{tatarski_wave_1967} se define la función de correlación de forma general para $f(t)$ como:

\begin{equation}\label{ec:correlation_2}
    B_f(t_1, t_2) = \overline{[f(t_1) - \overline{f(t_2)}][f^*(t_1) - \overline{f^*(t_2)}]}
\end{equation}

Para un proceso estacionario en el tiempo (su promedio temporal es constante $\overline{f(t)} = \text{const}$) la función de correlación dependerá unicamente de la diferencia temporal $t_1 - t_2$. Utilizando esto se puede simplificar \ref{ec:correlation_2} a:

\begin{equation}
    B_f(t_1 -t_2) = \overline{f(t_1)f^*(t_2)}
\end{equation}

Para facilitar los cálculos, a menudo se escogen funciones de correlación que involucran exponenciales, por ejemplo proporcionales a $e^{r/a}$ (\cite{chernov_wave_1967}) para la correlación de la posición. Estas funciones no caracterizan por completo algunas situaciones físicas. 

Para analizar turbulencia se usa en muchas ocasiones el valor cuadrático medio de la diferencia de una función evaluada en dos puntos, lo que se conoce como la función de estructura \cite{sasiela_electromagnetic_2007}. Mientras que la función de correlación es útil para procesos de alguna forma estacionarios, la función de estructura caracteriza un proceso aleatorio
con incrementos estacionarios. Para un fluido turbulento como el aire de velocidad $v$ se define como:

\begin{equation}
    D_v(\vec{r}) = \left\langle{[v(\vec{r}) - \vec{v}(\vec{r}+\vec{a})]^2}\right\rangle
\end{equation}

Donde los paréntesis angulados indican promedio de todo el conjunto. 

En el subrango inercial (que supone escalas de 1mm a 10m), donde la disipación no es importante se puede realizar la suposición de que la función de estructura depende unicamente de la magnitud de separación $r$ y de la energía depositada por segundo y unidad de masa $\varepsilon$ (\cite{kolmogorov_local_1941}):

\begin{equation}
    D_v(\vec{r}) = D_v(r) = f(r,\varepsilon)
\end{equation}

Como las unidades de $\varepsilon$ son $\frac{\text{J}}{\text{kg}\cdot \text{s}} = \frac{\text{m}^2}{\text{s}^3}$, la única combinación posible para formar la función de correlación (que tiene unidades de velocidad al cuadrado) debe tener la forma: 

\begin{equation}\label{ec:estructura_vel_2}
    D_v(r) = e^2(\varepsilon r)^{\frac{2}{3}}
\end{equation}

Donde $e$ es una constante. Debido a la característica aleatoria del campo de velocidad, esta distribución solo tiene sentido si la distancia considerada es mucho más grande que una cierta escala interna del fluido $r\gg L_{\text{in}}$. Esta ley se denomina ley de dos tercios y entra dentro de la clase de turbulencia isotrópica y homogénea desarrollada por Kolmogorov.

Fuera del subrango inercial y para separaciones muy pequeñas ($r \ll L_{\text{in}}$) la disipación domina y se tiene flujo laminar. De esta forma se puede expandir $D_v(r)$ en serie, y despreciando los términos superiores se obtiene: 

\begin{equation}\label{ec:estructura_vel_3}
    D_v(r) = b^2r^2
\end{equation}

Esto corresponde a un flujo laminar, donde la turbulencia no se ha desarrollado. En ocasiones es necesario trabajar en el espacio de Fourier. Para el coeficiente de refracción, la función de estructura se define como:

\begin{equation}
    D_n(\vec{r}) = \left\langle \left|n(\vec{a}) - n(\vec{a} + \vec{r})\right|^2 \right\rangle
\end{equation}

Se puede demostrar con el concepto de aditivos pasivos (\cite{tatarski_wave_1967}) que esta función sigue una expresión con la misma forma que la función de estructura de la velocidad \ref{ec:estructura_vel_2} y \ref{ec:estructura_vel_3}:

\begin{equation}
    D_n(\mathbf{r}) = C_n^2 r^{2/3}, \quad r \gg L_i
\end{equation}

\begin{equation}
    D_n(\mathbf{r}) = C_n^2 \frac{b^2}{e^2} r^2, \quad r \ll L_i
\end{equation}

Donde $C_n$ expresa la fuerza de la turbulencia. Realizando la transformación al espacio de Fourier (\cite{tatarski_wave_1967}, \cite{sasiela_electromagnetic_2007}) es posible encontrar que el espectro tridimensional $\Psi(k)$ es:

\begin{equation}
    \Psi = 0.0330054 \cdot C_n^2k^{-11/3}
\end{equation}

Donde $k$ es el número de onda. Esta clase de espectro se conoce como espectro de Kolmogorov.

\subsection{Resultados teóricos}\label{sec:resultados_teoricos}

Los métodos teóricos utilizados para modelizar las ondas electromagnéticas que se propagan a través de la turbulencia, en particular la atmosférica, se centran principalmente en los efectos de las fluctuaciones del índice de refracción causadas por la mezcla turbulenta (\cite{tatarski_wave_1967}). Estas fluctuaciones pueden provocar fenómenos como la dispersión y fluctuaciones en los parámetros de las ondas, como la amplitud y la fase. Los enfoques teóricos suelen empezar por describir el medio turbulento utilizando conceptos de la teoría de la turbulencia, como las funciones de estructura y el análisis espectral, y después aplican las ecuaciones de Maxwell y la ecuación de ondas para modelizar la propagación de las ondas.

Un punto de partida fundamental son los resultados de la teoría de la turbulencia, especialmente para describir las propiedades estadísticas del flujo turbulento y las fluctuaciones del índice de refracción resultantes. La teoría de Kolmogorov de la turbulencia localmente isótropa se utiliza con frecuencia. Esta teoría introduce el concepto de funciones de estructura para describir campos aleatorios, en particular cuando se trata de funciones aleatorias generalizadas que son más generales que las estacionarias. La microestructura del índice de refracción en un flujo turbulento, que es crucial para la propagación de ondas, se describe utilizando funciones de estructura y funciones espectrales. La densidad espectral de estas fluctuaciones puede representarse y utilizarse en los modelos de propagación de ondas. La representación de campos aleatorios mediante expansiones espectrales generalizadas es una herramienta valiosa para resolver formalmente problemas e interpretar resultados en la teoría de la propagación de ondas en un medio turbulento.

Se pueden discutir dos enfoques teóricos principales para modelar la propagación de ondas:

\begin{itemize}
    \item Enfoque de óptica geométrica: Este método es aplicable para ondas cortas, concretamente cuando la longitud de onda es mucho menor que la escala interna de la turbulencia. Comienza considerando las ecuaciones de Maxwell para la propagación de ondas en un medio no homogéneo. Para las ondas cortas, en las que las dimensiones geométricas de las inhomogeneidades son mucho mayores que la longitud de onda, pueden despreciarse algunos términos, lo que conduce a una ecuación de onda escalar. El campo de ondas $u$ suele representarse de la forma $u = A e^{iS}$, donde $A$ es la amplitud y $S$ es la fase. Dado que las fluctuaciones del índice de refracción en casos reales son muy pequeñas, puede aplicarse la teoría de perturbaciones para resolver las ecuaciones resultantes de las fluctuaciones de amplitud y fase. El campo aleatorio localmente isótropo de las fluctuaciones del índice de refracción puede representarse mediante integrales estocásticas y expansiones espectrales, que luego se utilizan para calcular las fluctuaciones de amplitud y fase. Este enfoque arroja resultados que muestran que las fluctuaciones de amplitud y fase son proporcionales al cuadrado de la frecuencia y de la distancia recorrida en el medio turbulento, resultados que no dependen de la forma específica de la función espectral o de estructura de las inhomogeneidades. Sin embargo, la teoría basada en la óptica geométrica sólo es válida para distancias limitadas, concretamente cuando $\sqrt{\lambda  L}$ es mucho menor que la escala interna de la turbulencia, o equivalentemente, cuando la distancia $L$ es mucho menor que una distancia crítica $L_{cr}$.
    \item Ecuación de onda con métodos de perturbación: Cuando no se cumplen las condiciones de la óptica geométrica, en particular para distancias mayores en las que la difracción de la onda por las inhomogeneidades del índice de refracción se hace significativa, es necesario partir de la ecuación de onda completa. Suponiendo que el campo de índice de refracción $n(r)$ es $n(r) = 1 + n_1(r)$ donde $|n_1(r)| \ll 1$, se aplica la teoría de perturbaciones a la ecuación de onda. Se mencionan dos formas del método de perturbación en relación con la propagación de ondas en un medio con pequeñas inhomogeneidades: la forma usual y la forma modificada de Rytov.
\end{itemize}

\subsubsection{Enfoque de óptica geométrica}

Siguiendo \cite{chernov_wave_1967}, se desarrollará la ecuación de rayos en presencia de inhomogeneidades aleatorias. Se considerará que la escala de estas inhomogeneidades es grande comparado con la longitud de onda $\lambda$.

La ecuación de rayos puede deducirse a partir del principio de Fermat, que establece que el tiempo empleado por la luz en propagarse debe ser mínimo. De esta manera se debe minimizar: 

\begin{equation}
    S = \int_A^B\frac{d\sigma}{c} = \int_A^B n(x, y, z)d\sigma
\end{equation}

$n(x, y, z) = \frac{c}{c_{\text{medio}}}$ es el indice de refracción. Además $A$ y $B$ representan el punto final e inicial respectivamente y $d\sigma$ representa un segmento infinitesimal de la curva descrita. Introduciendo la parametrización $x = x(u)$, $y = y(u)$, $z = z(u)$, se puede expresar $d\sigma = \sqrt{\left(\frac{dx}{du}\right)^2 + \left(\frac{dy}{du}\right)^2 + \left(\frac{dz}{du}\right)^2}du$ o denotando la diferenciación por $u$ con un punto $d\sigma = \sqrt{\dot x^2 + \dot y^2 + \dot z^2}du$ . Tomando también los puntos $A = u_1$ y $B = u_2$ como constantes el funcional a minimizar es: 

\begin{equation}\label{ec:accion_optica}
    S = \int_{u_1}^{u_2}n(x, y, z)\sqrt{\dot x^2 + \dot y^2 + \dot z^2}du
\end{equation}

De esta forma es posible aplicar las ecuaciones de Euler-Lagrange al problema para encontrar la curva que minimiza \ref{ec:accion_optica}, que se puede llamar acción óptica de manera análoga a la acción de la mecánica clásica. Siguiendo con la analogía, el equivalente del lagrangiano es $F = n(x, y, z)\sqrt{\dot x^2 + \dot y^2 + \dot z^2}$, el cual debe seguir cada una de las ecuaciones de Euler $\frac{d}{du}\frac{\partial F}{\partial q_i} - \frac{\partial F}{\partial q_i}$, con $q_i = x, z, y$. De esta forma se obtienen tres ecuaciones con un alto grado de simetría, por ejemplo la ecuación para $x$ queda:

\begin{equation}
    \frac{d}{du}\left(\frac{n\dot x}{\sqrt{\dot x^2 + \dot y^2 + \dot z^2}}\right) - \sqrt{\dot x^2 + \dot y^2 + \dot z^2}\frac{\partial n}{\partial n} = 0
\end{equation}

Estas ecuaciones pueden ser simplificadas introduciendo un vector unitario $\vec s$ tangente al rayo, con componentes $s_i = \frac{d q_i}{d \sigma} = \frac{dq_i /du}{d\sigma/ du} =  \frac{\dot q_i}{\sqrt{\dot x^2 + \dot y^2 + \dot z^2}}$. De esta manera las ecuaciones quedan simplificadas a: 

\begin{equation}
    \frac{d(ns_i)}{d\sigma} - \frac{\partial n}{\partial q_i} = 0
\end{equation}

Estas tres ecuaciones pueden ser unidas en una sola ecuación vectorial: 

\begin{equation}\label{ec:rayo_general}
    \frac{d(n\vec s)}{d \sigma} - \nabla n = 0
\end{equation}

Si el índice de refracción se da como una función de las coordenadas, la ecuación \ref{ec:rayo_general} junto con $\vec s = \frac{d \vec r}{d \sigma}$ permite encontrar las trayectorias de un rayo dadas condiciones iniciales. 

Para aplicar métodos perturbativos a \ref{ec:rayo_general}, el indice de refracción se expandirá como $n(x, y, z) = 1+ \mu(x, y, z)$, con $|\mu| \ll 1$. Con esto la ecuación puede ser reescrita como 

\begin{equation}\label{ec:rayo_perturbado}
    \frac{d(n\vec s)}{d \sigma} - \nabla \mu = 0
\end{equation}

Es posible calcular la desviación cuadrática media $\overline{\epsilon^2}$ del ángulo $\epsilon$ de un rayo con respecto a su dirección original tras viajar una distancia $\Delta \sigma$. Se debe escoger el camino lo suficientemente grande para ser mayor que la distancia de correlación del indice de refracción, pero lo suficientemente pequeño como para que la desviación sea pequeña. Integrando \ref{ec:rayo_perturbado} en el camino:

\begin{equation}
    n'\vec{s'} - n\vec{s} = \int_{0}^{\Delta \sigma} \nabla \mu d\sigma
\end{equation}

La desviación del rayo dependerá unicamente del valor del indice de refracción en camino recorrido, no de su valor en los puntos inicial y final. Por ello, es posible tomar los índices como su valor medio de la unidad. Se obtiene:

\begin{equation}
    \vec{s'} - \vec{s} = \int_{0}^{\Delta \sigma} \nabla \mu d\sigma
\end{equation}

Elevando ambos lados al cuadrado y utilizando propiedades vectoriales de $\vec{s}$ junto con ángulos pequeños se puede obtener:

\begin{equation}
    (\vec{s'} - \vec{s})^2 = 2(1 - \vec{s'}\cdot \vec{s}) = 2(1-\cos(\epsilon)) \backsim\epsilon^2
\end{equation}

Con esto se puede obtener: 

\begin{equation}
    \epsilon^2 = \left(\int_0^{\Delta \sigma}\nabla \mu d\sigma\right)\cdot\left(\int_0^{\Delta \sigma}\nabla \mu d\sigma\right) = \int_0^{\Delta \sigma} \int_0^{\Delta \sigma}\nabla_1  \nabla_2(\mu_1 \mu_2) d\sigma_1 d\sigma_2
\end{equation}

Tomando el promedio temporal y utilizando la función de correlación: 

\begin{equation}
    \overline{\epsilon^2} = \overline{\mu^2}\int_0^{\Delta \sigma} \int_0^{\Delta \sigma} \nabla_1  \nabla_2 N(x_1-x_2, y_1-y_2, z_1-z_2)d\sigma_1 d\sigma_2
\end{equation}

Como la curvatura del rayo es pequeña, es posible reemplazar el camino de integración por una linea recta:

\begin{equation}
    \overline{\epsilon^2} = \overline{\mu^2}\int_0^{\Delta \sigma} \int_0^{\Delta \sigma} \nabla_1  \nabla_2 N(x_1-x_2, y_1-y_2, z_1-z_2)dr_1 dr_2
\end{equation}

Para simplificar, se puede realizar un cambio a coordenadas relativas $x = x_1 - x_2$, $y = y_1 - y_2$, $z = z_1 - z_2$ y $r = r_1 - r_2$ y coordenadas de centro de masa: $x_0 = \frac{1}{2}(x_1+x_2)$, $y_0 = \frac{1}{2}(y_1+y_2)$, $z_0 = \frac{1}{2}(z_1+z_2)$ y $r_0 = \frac{1}{2}(r_1+r_2)$. Con ello queda:

\begin{equation}
    \nabla_1  \nabla_2 N(x_1-x_2, y_1-y_2, z_1-z_2) = - \nabla^2N(x, y, z)
\end{equation}

Junto con $dr_1dr_2 = dr_0dr$. Teniendo en cuenta que la función de correlación debe, por análisis dimensional, depender de una longitud de correlación $a$. Suponiendo esta constante como $\Delta \sigma \gg a$, los límites de integración en $r$ será el conjunto de los números reales. Con todo esto queda: 

\begin{equation}
    \overline{\epsilon^2} = -\overline{\mu^2}\int_0^{\Delta \sigma} dr_0\int_{-\infty}^{\infty}\nabla^2N(x, y, z)dr
\end{equation}

Teniendo en cuenta que $N$ es una función par finalmente e integrando en la coordenada $r_0$ se obtiene:

\begin{equation}
    \overline{\epsilon^2} = -2\overline{\mu^2}\Delta \sigma \int_0^{\infty}\nabla^2N(x, y, z)dr
\end{equation}

Para un medio estadísticamente isotrópico, la función de correlación dependerá unicamente de $r =\sqrt{x^2+y^2+z^2}$ y el laplaciano será más simple en coordenadas esféricas:

\begin{equation}
    \overline{\epsilon^2} = -2\overline{\mu^2}{\Delta \sigma}\int_0^{\infty}\frac{1}{r^2}\frac{\partial}{\partial r} \left( r^2\frac{\partial N(r)}{\partial r}\right) dr
\end{equation}

Utilizando $N(r) = e^{-\frac{r^2}{a^2}}$ se puede encontrar:

\begin{equation}
    \overline{\epsilon^2} = 4 \sqrt{\pi} \Delta \sigma \frac{\overline{\mu^2}}{a}
\end{equation}

\subsubsection{Ecuación de onda con métodos perturbativos}

El problema a considerar puede formularse como sigue. Una onda electromagnética monocromática plana incide sobre un volumen $V$ de un medio turbulento; debido a la mezcla turbulenta dentro del volumen $V$, aparecen fluctuaciones irregulares del índice de refracción, que dispersan la onda electromagnética incidente. Se requiere encontrar la densidad media de la energía dispersada en una dirección dada. Para resolver el problema, supondremos que el campo de índice de refracción dentro del volumen V es una función aleatoria de las coordenadas y no depende del tiempo. Para esta sección se seguirá el capítulo 4 de \cite{tatarski_wave_1967}.

En el volumen del medio turbulento, los campos electromagnéticos vendrán descritos por las ecuaciones de Maxwell en la materia \ref{ec:maxwell_materia_div} y \ref{ec:maxwell_materia_rot}. Para este problema se considerará una ausencia de carga y corriente libres ($\rho_f = 0$, $\mathbf{J}_f = 0$). Además el medio se considerará un dieléctrico lineal $\mathbf{D} = \epsilon \mathbf{E}$, $\mathbf{H} = \frac{1}{\mu}\mathbf{B}$.

Suponiendo que la conductividad del medio es cero, y que la permeabilidad magnética relativa es la unidad. Además se considerará una dependencia temporal de los campos dada por el factor $e^{-iwt}$. De esta manera los campos completos tomarán la forma $\mathbf{E}(\vec{r}, t) = \vec{E}(\vec{r})e^{-iwt}$ y $\mathbf{H}(\vec{r}, t) = \vec{H}(\vec{r})e^{-iwt}$. De esta forma las ecuaciones de Maxwell para las amplitudes de los campos tomarán la siguiente forma:

\begin{equation}\label{ec:maxwell_simple_materia_div}
    \nabla\cdot (\epsilon\vec{E}) = 0 \quad \nabla\cdot \vec{B} = 0 
\end{equation}

\begin{equation}\label{ec:maxwell_simple_materia_curl}
    \nabla\times\vec{E} = ik\vec{H}; \quad \nabla\times \vec{H} = -ik\epsilon\vec{E}
\end{equation}

Donde $k = \frac{w}{c}$ es el número de onda de la onda electromagnética y $\epsilon$ es la constante dieléctrica del material.

Tomando el rotacional de la primera ecuación en \ref{ec:maxwell_simple_materia_curl} y usando \ref{ec:maxwell_simple_materia_div}:

\begin{equation}\label{ec:amplitud_E_materia}
    \nabla(\nabla\cdot\vec{E}) - \nabla^2 \vec{E} = k^2 \epsilon\vec{E}
\end{equation}

Como en general $\nabla\cdot(u\vec{A}) = \vec{A}\cdot(\nabla u) + u\nabla\cdot\vec{A}$, se obtendrá:

\begin{equation}
    \nabla\cdot(\epsilon\vec{E}) = \vec{E}\cdot(\nabla\epsilon) + \epsilon\nabla\cdot\vec{E} = 0
\end{equation}

Reorganizando términos y usando la regla de la cadena queda:

\begin{equation*}
    \epsilon\nabla\cdot\vec{E} = -\vec{E}\cdot(\nabla\epsilon) \Rightarrow \nabla\cdot\vec{E} = -\vec{E}\cdot\left(\frac{1}{\epsilon}\nabla\cdot\epsilon\right) = -\vec{E} \cdot \nabla \ln \epsilon
\end{equation*}

Lo que significa que: 
\begin{equation}\label{ec:amp_E_div_epsilon}
    \nabla\cdot\vec{E} = -\vec{E} \cdot \nabla \ln \epsilon
\end{equation}

Al ser la permeabilidad magnética la unidad, se puede tomar $\epsilon = n^2$. Utilizando esto, y sustituyendo \ref{ec:amp_E_div_epsilon} en \ref{ec:amplitud_E_materia} se llega a:

\begin{equation}
    -2\nabla(\vec{E} \cdot \nabla \ln n) - \nabla^2 \vec{E} = k^2 n^2\vec{E}
\end{equation}

Para comparar esta ecuación con la ecuación de onda en el vacío, se pueden reorganizar los términos de la siguiente manera:

\begin{equation}\label{ec:amp_E_termino_extra}
     \nabla^2 \vec{E} + k^2 n^2\vec{E} + 2\nabla(\vec{E} \cdot \nabla \ln n) = 0
\end{equation}

Es posible ver que si se realiza la aproximación en la cual $\epsilon$ es una constante que no depende la posición, el sumando final $\nabla(\vec{E} \cdot \nabla \ln \epsilon)$ tenderá a cero, ya que el gradiente de un campo constante es cero en todos los puntos. De esta forma se recupera $\nabla^2 \vec{E} + k^2 n^2\vec{E} = 0$, cuya forma es la parte espacial de una ecuación de onda.

Se resolverá la ecuación \ref{ec:amp_E_termino_extra} utilizando un método perturbativo. Para ello se supondrá que las fluctuaciones de $n$ son pequeñas. De esta forma, asumiendo que el valor promedio de $n$ es $\overline{n}$ y sus fluctuaciones vienen denotadas por $n_1$ se tendrá que $|n-\overline{n}| \ll 1$. Para simplificar los cálculos se asumirá que $\overline{n} = 1$, pero esto no supone una pérdida de generalidad, ya que es posible recuperar el término $\overline{n}$ substituyendo $k$ por $\overline{n}k$. Sustituyendo $n = 1 + n_1$ en \ref{ec:amp_E_termino_extra} se obtiene:

\begin{equation}
     \nabla^2 \vec{E} + k^2 (1+n_1)^2\vec{E} + 2\nabla(\vec{E} \cdot \nabla \ln (1+n_1)) = 0
\end{equation}

Que se puede expandir a:

\begin{equation}
     \nabla^2 \vec{E} + k^2 \vec{E}= - 2\nabla(\vec{E} \cdot \nabla \ln (1+n_1)) - 2k^2n_1 \vec{E} - k^2n_1^2 \vec{E}
\end{equation}

Se desea obtener una solución de $\vec{E}$ en una serie $\vec{E} = \vec{E_0} + \vec{E_1} + \vec{E_2}+\cdots$ donde cada término tiene el mismo orden de las respectivas potencias de $n_1$. Para ello, primero se expandirá el término del logaritmo utilizando su serie: $\ln(1+n_1) = n_1-\frac{1}{2}n_1^2+ \cdots$. En este caso, debido a los ordenes con los que trabajar y por simplicidad, se tomará el primer término de esta serie, despreciando términos con $n_1^2$. Con esto queda: 

\begin{equation}\label{ec:amp_E_pert_1}
     \nabla^2 \vec{E} + k^2 \vec{E}= - 2\nabla(\vec{E} \cdot \nabla n_1) - 2k^2n_1 \vec{E}
\end{equation}

Substituyendo esta serie truncada a $\vec{E} = \vec{E_0} + \vec{E_1}$ en \ref{ec:amp_E_pert_1} se tiene:

\begin{equation}
    \nabla^2 \vec{E_0} + k^2 \vec{E_0} + \nabla^2 \vec{E_1} + k^2 \vec{E_1} = - 2\nabla(\vec{E_0} \cdot \nabla n_1) - 2\nabla(\vec{E_1} \cdot \nabla n_1) - 2k^2n_1 (\vec{E_0} + \vec{E_1})
\end{equation}

Para proceder se deben agrupar términos por orden de magnitud e igualarlos a cero. Se pueden distinguir dos grupos (con ordenes iguales al orden de $n_1^0$ y $n_1$):

\begin{equation}
    \nabla^2 \vec{E_0} + k^2 \vec{E_0} = 0
\end{equation}

\begin{equation}\label{ec:amp_E_pert_grupo2}
    \nabla^2 \vec{E_1} + k^2 \vec{E_1} = - 2k^2n_1\vec{E_0} - 2\nabla(\vec{E_0} \cdot \nabla n_1)
\end{equation}

La cantidad $\vec{E_0}$ representa la amplitud del campo eléctrico de la onda incidente. Suponiendo que la onda incidente es plana, se puede fijar $\vec{E_0}= \vec{A_0}e^{i\vec{k}\cdot\vec{r}}$. La cantidad $\vec{E_1}$ representa la amplitud del campo eléctrico de la onda dispersada. 

La solución de la ecuación $\nabla^2u + k^2 u = f(\vec{r})$ (ecuación de Helmholtz inhomogenia) viene dada por (ver subsección \ref{subsect:green} junto con las referencias \cite{jackson_classical_1975}, \cite{morse_methods_1999}):

\begin{equation}\label{ec:solucion_hemholz_inhomogenia}
    u(\vec{r}) = \frac{1}{4\pi} \int_V f(\vec{r'}) \frac{\exp(ik |\vec{r}- \vec{r'}|)}{|\vec{r}- \vec{r'}|}dV'
\end{equation}

donde $\vec{r'}$ es el vector que recorre todo el volumen. Tomando el origen de coordenadas en el volumen, y suponiendo que $\vec{r}$ está muy alejado del mismo (relativamente al tamaño del volumen), es posible aproximar la cantidad $|\vec{r}- \vec{r'}|$ en primer orden como:

\begin{equation}
    |\vec{r}- \vec{r'}| = r -\hat{r} \cdot \vec{r'}
\end{equation}

Donde $\hat{r}$ es el vector unitario de $\vec{r}$. Esto se cumplirá si la dimensión del volumen $L$ satisface $\lambda r \gg L^2$. Además, en el denominador de \ref{ec:solucion_hemholz_inhomogenia} se puede reemplazar $|\vec{r}- \vec{r'}| = r$. Con esto queda:

\begin{equation}\label{ec:hemholz_inhomogenia_aprox}
    u(\vec{r}) = \frac{1}{4\pi} \frac{e^{ikr}}{r} \int_V f(\vec{r'}) e^{-ik\hat{r} \cdot \vec{r'}} dV'
\end{equation}

Se puede utilizar \ref{ec:hemholz_inhomogenia_aprox} para resolver \ref{ec:amp_E_pert_grupo2}, obteniendo:

\begin{equation}
    \vec{E_1} = \frac{k^2}{2\pi} \frac{e^{ikr}}{r} \int_V n_1(\vec{r'}) \vec{A_0}e^{i\vec{k}\cdot\vec{r'}} e^{-ik\hat{r} \cdot \vec{r'}} dV' + \frac{1}{2\pi} \frac{e^{ikr}}{r} \int_V \nabla(e^{i\vec{k}\cdot\vec{r'}} \vec{A_0}\cdot \nabla n_1(\vec{r'})) e^{-ik\hat{r} \cdot \vec{r'}} dV'
\end{equation}

Utilizando en teorema de Gauss, es posible llegar a la identidad $\int_V \nabla \phi dV'= \int_S \phi  d\vec{\sigma}$. Si aplicamos esta identidad a $\phi = uv$, siendo todos campos escalares se puede llegar a: 

\begin{equation}
    \int_V u\nabla v dV' = \int_S uvd\vec{\sigma} - \int_V v\nabla u dV'
\end{equation}

Lo que permite simplificar la segunda integral, ya que la integral sobre la superficie será cero si se escoge una superficie lo suficientemente alejada del volumen de integración. Aplicando el hecho de que $\nabla(e^{ik\hat{r} \cdot \vec{r'}}) = ik \hat{r} e^{-ik\hat{r} \cdot \vec{r'}}$ se puede obtener para la segunda integral:

\begin{equation}
    \int_V \nabla(e^{i\vec{k}\cdot\vec{r'}} \vec{A_0}\cdot \nabla n_1(\vec{r'})) e^{-ik\hat{r} \cdot \vec{r'}} dV' = ik \hat{r} \int_V (\vec{A}_0 \cdot \nabla n_1(\vec{r'})) e^{i\vec{k}\cdot\vec{r'}}e^{-ik\hat{r} \cdot \vec{r'}} dV'
\end{equation}

Finalmente se puede escribir el campo eléctrico como:

\begin{equation}\label{ec:solucion_E_pert}
    \vec{E_1} = \frac{k^2 e^{ikr}}{2\pi r} C_1 \vec{A}_0  + \frac{ike^{ikr}}{2\pi r}C_2\hat{r}
\end{equation}

Con

\begin{equation}
    C_1 =\int_V n_1(\vec{r'}) e^{i\vec{k}\cdot\vec{r'}} e^{-ik\hat{r} \cdot \vec{r'}} dV' \quad C_2=ik \hat{r} \int_V (\vec{A}_0 \cdot \nabla n_1(\vec{r'})) e^{i\vec{k}\cdot\vec{r'}}e^{-ik\hat{r} \cdot \vec{r'}} dV'
\end{equation}

Para los cálculos de energía se asumirá que el campo es puramente transversal, eliminando el segundo término de la ecuación \ref{ec:solucion_E_pert}.

Para calcular la densidad del flujo de la energía dispersada es necesario calcular el vector de Poynting (o mejor dicho su promedio temporal en un periodo de oscilación) $\vec{S} = \frac{1}{2} \text{Re} \left(\vec{E}_1 \times \vec{H}_1^*\right)$. Para ello es necesario encontrar $\vec{H}$, lo que se puede hacer con la primera de las ecuaciones en \ref{ec:maxwell_simple_materia_curl}:

\begin{equation}\label{ec:solucion_H_pert}
    \vec{H}_1 = \frac{k^2 C_1}{2\pi i k} \nabla\times \left(\frac{e^{ikr}}{r} \vec{A}_0\right) = \frac{k^2 C_1}{2\pi i k} \vec{A}_0 \times  \nabla \left(\frac{e^{ikr}}{r}\right)
\end{equation}

$$= \frac{k^2 C_1}{2\pi i k} \left(\frac{ike^{ikr}}{r} - \frac{e^{ikr}}{r^2}\right) \hat{r} \times \vec{A}_0 \sim \frac{k^2 C_1 e^{ikr}}{2\pi r} \hat{r} \times \vec{A}_0$$

Donde se desprecia el término $\frac{e^{ikr}}{r^2}$ por decaer muy rápido. Sustituyendo \ref{ec:solucion_E_pert} y \ref{ec:solucion_H_pert} en la fórmula para el vector de Poynting se obtiene: 

\begin{equation}
    \vec{S} = \frac{k^4}{16\pi r^2} C_1 C_1^* \vec{A}_0 \times (\hat{r} \times \vec{A}_0) = \frac{k^4}{16\pi r^2} C_1 C_1^* (\hat{r}(\vec{A}_0 \cdot \vec{A}_0) - \vec{A}_0(\hat{r} \cdot \vec{A}_0))
\end{equation}

La densidad de flujo energético en la dirección de $\vec{r}$ es:

\begin{equation}
    S_{r} = \vec{S} \cdot \hat{r} = \frac{k^4 C_1 C_1^*}{16\pi^2 r^2} (A_0^2 - (\hat{r} \cdot \vec{A}_0)^2) = \frac{k^4 A_0^2 \sin^2\chi}{16\pi^2 r^2} C_1 C_1^*
\end{equation}

Donde $\chi$ es el ángulo entre los vectores $\vec{A_0}$ y $\hat{r}$. Substituyendo la expresión para $C_1$ se obtiene:

\begin{equation}
    S_{\hat{r}} = \frac{k^4 A_0^2 \sin^2\chi}{16\pi^2 r^2} \iint_{V V} n_1(\vec{r}_1) n_1(\vec{r}_2) e^{i(\vec{k} - k\hat{r}) \cdot (\vec{r}_1 - \vec{r}_2)} dV_1 dV_2
\end{equation}

Esto significa que la cantidad $S_r$ es un campo aleatorio. Su valor medio es igual a:

\begin{equation}\label{ec:pointyn}
    \overline{S}_{r} = \frac{k^4 A_0^2 \sin^2\chi}{16\pi^2 r^2} \iint_{V V} \overline{n_1(\vec{r}_1) n_1(\vec{r}_2)} e^{i(\vec{k} - k\hat{r}) \cdot (\vec{r}_1 - \vec{r}_2)} dV_1 dV_2
\end{equation}

Por lo tanto queda $S_m$ expresado en términos de la función de correlación espacial, tal y como se define en \cite{tatarski_wave_1967}, $B_f(\vec{r}_1 - \vec{r}_2) = \overline{n(\vec{r}_1)n(\vec{r}_2)}$. Realizando el cambio de variable $2\vec{r} = \vec{r}_1 + \vec{r}_2$ y $\vec{\rho} = \vec{r}_1 - \vec{r}_2$, es posible simplificar \ref{ec:pointyn} realizando la integral con respecto a $r$ lo que añade un factor de $V$:

\begin{equation}\label{ec:pointyn_2}
    \overline{S}_{r} = \frac{k^4 VA_0^2 \sin^2\chi }{16\pi^2 r^2}\int_V B_f(\vec{\rho}) e^{i(\vec{k} - k\hat{r}) \cdot \vec{\rho}} dV_\rho
\end{equation}

Desde este punto se deben seguir métodos espectrales en la función de correlación para seguir transformando la expresión. En específico si se expresa la función de correlación en función de su espectro en el espacio de Fourier $\Phi_n(\vec\kappa)$ se puede llegar a:

\begin{equation}
    \overline{S}_{\vec{m}} = \frac{k^4 V A_0^2 \sin^2\chi}{r^2} \tilde{\Phi}_n(\vec{k} - k\vec{m})
\end{equation}

Donde $ \tilde{\Phi}_n$ indica un promedio espacial en un volumen en el espacio de Fourier. De esta manera se puede obtener el flujo de energía de la onda perturbada.

Es posible calcular tambien el llamado nivel de fluctuaciones en escala logaritmica. Suponiendo que las dimensiones geométricas de todas las inhomogeneidades en la distribución espacial del índice de refracción son mucho mayores que la longitud de onda $\lambda$ (es decir, que $\lambda \ll l_0$, donde $l_0$ es la escala interna de la turbulencia), es posible despreciar el último término de la ecuación \ref{ec:amp_E_termino_extra}. Así, la propagación de ondas cortas ($\lambda \ll l_0$) en un medio no homogéneo se describe mediante la ecuación:

\begin{equation}\label{ec:E_medio_no_homogeneo}
    \nabla^2 \vec{E} + k^2 n^2(\vec{r})\vec{E} = 0
\end{equation}

La ecuación vectorial \ref{ec:E_medio_no_homogeneo} se reduce a tres ecuaciones escalares:

\begin{equation}\label{ec:u_basica}
    \nabla^2 u + k^2 n^2(\vec{r})u = 0
\end{equation}

donde $u$ representa cada uno de los componentes de $\vec{E}$. Se tomará $u = A e^{iS}$, donde $A$ es la amplitud y $S$ la fase. En el análisis completo (referise a \cite{tatarski_wave_1967}), se expande $\ln A = \ln A_0 + \chi$, donde se define $\chi$ como el nivel de fluctuaciones de la amplitud o amplitud logarítmica (log-amplitud). Utilizando métodos espectrales para trabajar con las inhomogeneidades y expandiendo \ref{ec:u_basica} con métodos perturbativos es posible llegar a la expresión:

\begin{equation}\label{ec:log_amplitud}
    \overline{\left(\log \frac{A}{A_0}\right)^2} = 2.46 C_n^2 L^3l_0^{-7/3}
\end{equation}

Donde $L$ es la distancia recorrida por la onda, $l_0$ es la escala interna de la turbulencia y $C_n$ expresa la fuerza de la turbulencia. Así, la fluctuación cuadrática media del logaritmo de la amplitud depende de las dimensiones de las menores inhomogeneidades del índice de refracción (en la escala interna de la turbulencia $l_0$) y es proporcional a la característica $C_n$ de la función de estructura de las fluctuaciones del índice de refracción.

\section{Resultados numéricos}

\subsection{Presentación del algoritmo FDTD}

El algoritmo FDTD (de las siglas en inglés Finite Diference Time Domain, que se puede traducir como Diferencia Finita Dominio de Tiempo) propuesto por primera vez por Kane Yee en 1966 (\cite{kane_yee_numerical_1966}) emplea diferencias centrales de segundo orden. Siguiendo \cite{taflove_computational_2010}, \cite{schneider_understanding_2025} el algoritmo será presentado y ejemplificado. Los pasos del algoritmo pueden resumirse como:

\begin{enumerate}
    \item Sustituir todas las derivadas de las leyes de Ampère y Faraday por diferencias finitas. Discretizar el espacio y el tiempo de forma que los campos eléctricos y magnéticos estén escalonados tanto en el espacio como en el tiempo.
    \item Resolver las ecuaciones en diferencias resultantes para obtener ecuaciones de actualización que expresen los campos futuros (desconocidos) en términos de los campos pasados (conocidos).
    \item Obtener los campos magnéticos en el futuro.
    \item Obtener los campos eléctricos en el futuro
    \item Repetir los dos pasos anteriores hasta obtener los campos a lo largo de la duración deseada.
\end{enumerate}

Primero el algoritmo será presentado de forma simplificada en una dimensión espacial y con permitividad y permeabilidad constantes.

En ausencia de cargas y corrientes externas ($\rho_f = 0$, $\mathbf{J}_f = 0$) las ecuaciones de Maxwell en la materia \ref{ec:maxwell_materia_div} y \ref{ec:maxwell_materia_rot} se reducen a:

\begin{equation}\label{ec:maxwell_materia_div_no_libre}
    \nabla\cdot \mathbf{D} = 0 \quad \nabla\cdot \mathbf{B} = 0 
\end{equation}

\begin{equation}\label{ec:maxwell_materia_rot_no_libre}
    \nabla\times\mathbf{E} = - \frac{\partial\mathbf{B}}{\partial t}; \quad \nabla\times \mathbf{H} = \frac{\partial\mathbf{D}}{\partial t}
\end{equation}

El algoritmo se desarrolla en torno a las ecuaciones que involucran un rotacional y una derivada en el tiempo (las leyes de Ampère y Faraday)-

Suponiendo que el campo eléctrico solo tiene componente en $z$, la primera ecuación de \ref{ec:maxwell_materia_rot_no_libre} (la ley de Faraday) puede simplificarse expandiendo el rotacional, teniendo en cuenta que $\textbf{E} = E_x \hat{x} + E_y \hat{y} + E_z \hat{z}$:

\begin{equation}\label{ec:faraday_1D}
    -\mu \frac{\partial \textbf{H}}{\partial t} = \nabla \times \textbf{E} = -\frac{\partial E_z}{\partial x} \hat y
\end{equation}

Expandiendo el el vector $\textbf{H} = H_x \hat{x} + H_y \hat{y} + H_z \hat{z}$ de la ecuación \ref{ec:faraday_1D} se puede ver que $\textbf{H}$ sólo tendrá componente en la dirección $y$. Sabiendo esto la ley de Ampère queda: 

\begin{equation}
    \epsilon \frac{\partial \textbf{E}}{\partial t} = \nabla \times \textbf{H} = -\frac{\partial H_y}{\partial x} \hat z
\end{equation}

De esta manera, las ecuaciones escalares a discretizar son:

\begin{equation}
    \mu \frac{H_y}{\partial t} = \frac{\partial E_z}{\partial x}
\end{equation}

\begin{equation}
    \epsilon \frac{E_z}{\partial t} = \frac{\partial H_y}{\partial x}
\end{equation}

La primera ecuación da la derivada temporal del campo magnético en términos de la derivada espacial del campo eléctrico. A la inversa, la segunda ecuación da la derivada temporal del campo eléctrico en términos de la derivada espacial del campo magnético, del campo eléctrico en función de la derivada espacial del campo magnético. Como se verá, la primera ecuación se utilizará para hacer avanzar el campo magnético en el tiempo, mientras que la segunda se utilizará para hacer avanzar el campo eléctrico. Un método en el que se hace avanzar una cantidad y luego otra, y luego se repite el proceso, se conoce como método "leapfrog" (del inglés "salto de rana").

El siguiente paso es reemplazar las derivadas por diferencias finitas centradas. Los puntos discretos en el espacio y el tiempo se denotarán de la siguiente manera:

\begin{equation}
    E_z(x, t) \rightarrow (E_z)^q_m 
\end{equation}

De manera similar se utilizará $(H_y)^q_m$ para el campo magnético. Los puntos discretos de los campos se colocarán en el espacio siguiendo la cuadricula de Yee. De esta forma, el campo eléctrico estará en los puntos enteros de la cuadrícula y el campo magnético en los puntos semienteros. Una representación visual de esta cuadrícula puede verse en \ref{fig:yee_grid_1D}.

\begin{figure}[h]
    \centering
    \includegraphics[width=0.7\linewidth]{figures/yee_grid.png}
    \caption{Representación visual de la rejilla de Yee en una dimensión}
    \label{fig:yee_grid_1D}
\end{figure}

Reemplazando la derivada temporal por una diferencia finita centrada y la derivada espacial por una diferencia finita adelantada, la ley de Faraday se puede discretizar en $\left((m + \frac{1}{2})\Delta x, q\Delta t\right)$:

\begin{equation}
    \mu \frac{(H_y)^{q+\frac{1}{2}}_{m+\frac{1}{2}} - (H_y)^{q-\frac{1}{2}}_{m+\frac{1}{2}}}{\Delta t} = \frac{(E_z)^q_{m+1}- (E_z)^q_{m}}{\Delta x}
\end{equation}

Que se puede resolver para encontrar $(H_y)^{q+\frac{1}{2}}_{m+\frac{1}{2}}$:

\begin{equation}\label{ec:update_H_1D}
    (H_y)^{q+\frac{1}{2}}_{m+\frac{1}{2}} = (H_y)^{q-\frac{1}{2}}_{m+\frac{1}{2}} + \frac{\Delta t}{\mu \Delta x} ((E_z)^q_{m+1}- (E_z)^q_{m})
\end{equation}

Procediendo de manera similar en $(m\Delta x, (q+ \frac{1}{2})\Delta t)$ para la ley de Ampère:

\begin{equation}
    \epsilon \frac{(E_z)^{q+1}_{m} - (E_z)^{q}_{m}}{\Delta t} = \frac{(H_y)^{q + \frac{1}{2}}_{m+\frac{1}{2}} - (H_y)^{q+ \frac{1}{2}}_{m-\frac{1}{2}}}{\Delta x}
\end{equation}

Y resolviendo para encontrar el campo eléctrico en el siguiente paso temporal:

\begin{equation}\label{ec:update_E_1D}
    (E_z)^{q+1}_{m} = (E_z)^{q}_{m} + \frac{\Delta t}{\epsilon \Delta x}\left((H_y)^{q + \frac{1}{2}}_{m+\frac{1}{2}} - (H_y)^{q+ \frac{1}{2}}_{m-\frac{1}{2}}\right)
\end{equation}

Estos pasos de resolución para encontrar $(E_z)^{q+1}_{m}$ y $(H_y)^{q+\frac{1}{2}}_{m+\frac{1}{2}}$ son precisamente los que suponen la permeabilidad y permitividad constantes. A la hora de generalizar, se deberá tomar $\epsilon$ y $\mu$ como tensores y se deberá encontrar su inverso.

A menudo es conveniente representar los coeficientes de actualización $\frac{\Delta t}{\epsilon \Delta x}$ y $\frac{\Delta t}{\mu \Delta x}$ en términos de la relación entre lo lejos que puede propagarse la energía en un solo paso temporal y el paso espacial. La velocidad a la que puede viajar la energía electromagnética es la velocidad de la luz en el espacio libre $c = \frac{1}{\sqrt{\epsilon_0 \mu_0}}$ y, por tanto la distancia máxima que puede recorrer la energía en un paso temporal es $c\Delta t$ (el símbolo c se reservará para la velocidad de la luz en el espacio libre). El cociente $S_c = \frac{c\Delta t}{\Delta x}$ suele denominarse el número de Courant (introducido por primera vez en \cite{courant_partial_1956}). Desempeña un papel importante en la determinación de la estabilidad de una simulación. Siendo $\mu = \mu_r \mu_0$ y $\epsilon = \epsilon_r \epsilon_0$, los coeficientes pueden escribirse:

\begin{equation}
    \frac{\Delta t}{\mu \Delta x} = \frac{S_c}{\mu c} = \frac{S_c \sqrt{\epsilon_0 \mu_0}}{\mu} = \frac{S_c \sqrt{\epsilon_0}}{\mu_r \sqrt{\mu_0}} = \frac{S_c }{\mu_r \eta_0}
\end{equation}

Junto con $\frac{\Delta t}{\epsilon \Delta x} = \frac{S_c \eta_0}{\epsilon_r}$. $\eta_0 = \sqrt \frac{\epsilon_0}{\mu_0}$ es la impedancia característica del espacio vacío, medida en Ohmnios. Al ser $S_c$, $\mu_r$ y $\epsilon_r$ adimensionales, $\eta_0$ representará todas las unidades del problema. Por ahora la impedancia se considera constante.

En las simulaciones FDTD existen restricciones en cuanto al tamaño del paso temporal. Si es demasiado grande el algoritmo produce resultados inestables (es decir, los números obtenidos carecen de sentido y tienden rápidamente al infinito). En esta fase no vamos a considerar un análisis riguroso de la estabilidad. Sin embargo, pensando en la forma en que se propagan los campos en una cuadrícula FDTD, parece lógico que la energía no pueda propagarse más allá de un paso espacial por cada paso temporal, es decir, $c\Delta t < \Delta x$. Esto se debe a que en el algoritmo FDTD cada nodo solo afecta a sus vecinos más cercanos. En un ciclo completo de actualización de los campos, lo más lejos que podría propagarse una perturbación es un paso espacial. Resulta que la relación óptima para el número de Courant (en términos de minimizar los errores numéricos) es también la relación máxima. Por lo tanto, para las simulaciones unidimensionales consideradas inicialmente:

\begin{equation}
    S_c = \frac{c\Delta t}{\Delta x} = 1
\end{equation}

Para poder completar la simulación, es necesario considerar las condiciones de contorno del problema y las condiciones iniciales. Al existir un número finito de puntos, no es posible usar las ecuaciones de actualización para los puntos iniciales y finales, por lo que se deben tratar aparte. En este caso sencillo, se resolverán ambas cuestiones al mismo tiempo: para el campo eléctrico en $x = 0$ se introducirá una fuente que decae exponencialmente; y para el campo magnético en el término del eje x se mantendrá nulo. 

En el caso de dos y tres dimensiones se deben generalizar las ecuaciones de actualización de ambos campos \ref{ec:update_E_1D} y \ref{ec:update_H_1D} mediante la discretización del operador rotacional con todas sus derivadas. Si la permeabilidad y permitividad se permiten variar, se debe encontrar su inverso para poder resolver las ecuaciones de forma eficiente. 

El lenguaje de programación Python junto con la librería de FDTD creada por Floris Laporte (\cite{laport_python_nodate}) fueron escogidos para realizar la mayor parte de la simulación. 

Es necesario discutir tambien las condiciones de contorno. Se mencionarán solo dos: condiciones de contorno periódicas y condiciones absorbentes. Las condiciones de contorno periódicas replican el valor de los campos en unas células en células del lado opuesto. Principalmente simulan periodicidad infinita. 

En simulaciones de FDTD, comúnmente se utilizan (\cite{taflove_computational_2010}) una clase de condición de contorno absorbente llamada Perfectly Matched Layer o PLM (que se puede traducir como capa perfectamente equiparada). La PML es una condición de contorno absorbente muy eficaz utilizada para simular la extensión de una red de cálculo hasta el infinito. Permite que las ondas numéricas escapen del espacio de cálculo con una reflexión insignificante. 

\subsection{Resultados en dos dimensiones}

Para simular el campo electromagnético utilizando el método FDTD, se debe proporcionar tanto la permitividad como la permeabilidad relativas del medio. Esto supone el punto que acopla o une ambos lados de la simulación: el campo electromagnético y el fluido. Para baja intensidades del campo electromagnético, es razonable asumir que el fluido no se ve afectado por el campo electromagnético. Además, debido a las escalas de tiempo de ambas partes de la simulación, (la velocidad de la luz es varios órdenes de magnitud mayor que la velocidad del sonido en cualquier medio) se puede asumir en la mayoría de casos que el fluido es estático con respecto a la propagación de ondas.

De esta forma se pueden esbozar tres pasos para la simulación: simular el fluido, transformar las variables del fluido en permeabilidad y permitividad, y por ultimo simular el campo electromagnético. Sin embargo, una simulación completa del fluido utilizando por ejemplo las ecuaciones de Navier-Stokes puede no ser estrictamente necesaria, ya que muchas de las características de la turbulencia pueden capturarse utilizando campos aleatorios.

Para obtener la permeabilidad y permitividad se puede trabajar con las variables macroscópicas del fluido. Derivar la permeabilidad y permitividad de propiedades microscópicas se discutirá más adelante. Suponiendo que el fluido no es magnético en las condiciones a trabajar, se puede tomar $\mu_r \approx 1$ y por lo tanto:

\begin{equation}
    n = \sqrt{\epsilon_r \mu_r} \approx \sqrt{\epsilon_r} \Rightarrow \epsilon_r = n^2
\end{equation}

Inicialmente y por sencillez, se modelará el índice de refracción como un campo aleatorio. Para informar el valor del indice y su distribución para condiciones comúnmente encontradas se pueden usar las fórmulas encontradas en \cite{ciddor_refractive_1996} y \cite{mathar_refractive_2007}. Una fórmula más simple puede encontrarse en \cite{sasiela_electromagnetic_2007}:

\begin{equation}
    n - 1 = 77.6\times10^{-6}(1 + 7.52\times10^{-3}/\lambda^2)\left(\frac{P}{T}\right)
\end{equation}

Donde $\lambda$ es la longitud de onda, $P$ es la presión en milibares y $T$ es la temperatura en grados Kelvin. Con esta fórmula se puede representar la permitividad en función de los parámetros de presión y temperatura. Una representación de esta función para un rango de presiones y temperaturas se puede ver en la figura \ref{fig:estimacion_perm_15}. Con esto se puede obtener que para longitud de onda $1.5 \cdot 10^{-5}$, esta estimación da un promedio de $\epsilon_r-1$ de 0.00054, con una desviación estándar de $1\cdot10^{-5}$. Estos valores representan un punto de partida a partir del cual modelar la permitividad como una variable aleatoria gaussiana. 

\begin{figure}[ht]
    \centering
    \includegraphics[width=0.7\linewidth]{figures/estimacion_perm_15.png}
    \caption{Siguiendo \cite{sasiela_electromagnetic_2007} se puede obtener un rango de valores para la permitividad del aire dadas la temperatura y la presión}
    \label{fig:estimacion_perm_15}
\end{figure}

En la simulación FDTD se establecerá una rejilla como se puede ver en la figura \ref{fig:setup1}. De esta manera se tendrá unas condiciones de contorno absorbentes, implementadas como PLMs, una fuente con perfil gaussiano, un detector y una zona con la permitividad modificada que representa aire turbulento. De esta manera se intenta modelar un problema de propagación de ondas similar a los vistos en la sección \ref{sec:resultados_teoricos}. Se ha escogido el espaciado de la rejilla como diez veces la longitud de onda de la fuente. La longitud de onda se ha tomado como $\lambda = 2\mu$m en todas las simulaciones.

\begin{figure}
    \centering
    \includegraphics[width=0.7\linewidth]{figures/setup_2d_1.png}
    \caption{Esquema visual de la rejilla a simular. Los ejes vienen en índices de la rejilla}
    \label{fig:setup1}
\end{figure}

Para crear la zona de permitividad modificada primero se genera una array con unas dimensiones determinadas tomando una muestra aleatoria de una gaussiana. Entonces, esta array se re-escala mediante una interpolación con splines de tercer orden a la resolución necesaria para encajar en la rejilla FDTD. La interpolación es manejada por la función zoom de SciPy. De esta manera se obtienen resultados como \ref{fig:perm_aleatoria}. Controlando las dimensiones iniciales y finales se puede ajustar el tamaño de las inhomogeneidades de la zona. En este caso las dimensiones iniciales eran de 20 por 10 y las finales de 400 por 200, por lo que al ser la longitud de onda diez espacios de rejilla la zona tendrá un tamaño de inhomogeneidades de 20 espacios de rejilla. 

\begin{figure}
    \centering
    \includegraphics[width=0.7\linewidth]{figures/perm_aletoria.png}
    \caption{Zona de $\epsilon_r$ aleatoria con inhomegenidades de tamaño aproximado de 20 espacios de rejilla}
    \label{fig:perm_aleatoria}
\end{figure}

Una vez simulada la rejilla durante los suficientes pasos temporales, la onda se propaga por todo el dominio llegando al detector. El número de pasos temporales tomados fueron 800. Los componentes de los campos finales pueden verse en \ref{fig:campos_fin}. El resultado de la energía en el dominio puede verse en \ref{fig:setup_2d_2}. 

\begin{figure}
    \centering
    \includegraphics[width=0.7\linewidth]{figures/setup_2d_2.png}
    \caption{Representación de la energía en la rejilla una vez finalizada la simulación. Los ejes vienen en indices de la rejilla}
    \label{fig:setup_2d_2}
\end{figure}

\begin{figure}
    \centering
    \includegraphics[width=0.8\linewidth]{figures/campos_fin.png}
    \caption{Componentes de los campos obtenidos tras 800 pasos temporales}
    \label{fig:campos_fin}
\end{figure}

La densidad de energía del campo electromagnético es (\cite{jackson_classical_1975}): 

\begin{equation}
    u = \frac{1}{2} \left(\epsilon E^2 + \frac{1}{\mu} B\right)
\end{equation}

Teniendo en cuenta que en general $c = \frac{1}{\sqrt{\epsilon_0 \mu_0}}$ y en el caso considerado $\mu = \mu_0$, se puede simplificar la expresión:

\begin{equation}
    u = \frac{1}{2} \left(\epsilon E^2 + \frac{1}{\mu} B\right) = \frac{1}{2} \epsilon \left(\epsilon_r E^2 + \frac{1}{\epsilon_0\mu_0} B\right) 
\end{equation}

Por otro lado en términos de $\vec H$:

\begin{equation}
    u = \frac{1}{2} \left(\epsilon E^2 + \mu H^2 \right) = \frac{1}{2} \left(\epsilon E^2 + \mu_0 H^2 \right) = \frac{1}{2} \epsilon_0 \left(\epsilon_r E^2 +  \frac{1}{\eta^2_0} H^2 \right)
\end{equation}

Donde $\eta_0 = \sqrt \frac{\epsilon_0}{\mu_0}$ es la impedancia característica del espacio vacío, medida en Ohmios. Como tiene un valor constante de $\eta_0 = 377.0$ será posible despreciar el campo magnético en el cálculo energético, al ser ambos campos de similar orden de magnitud.

Realizando la misma simulación sin la zona de permitividad modificada es posible encontrar una referencia con la que comparar los resultados obtenidos. Esto será de utilidad cuando se comparen estos resultados con los de la sección \ref{sec:resultados_teoricos}. Un resultado de muestra para para el ratio de densidad energética en el detector puede verse en \ref{fig:comparacion_2d}. 

\begin{figure}
    \centering
    \includegraphics[width=0.7\linewidth]{figures/comparacion_2d.png}
    \caption{Una posible muestra del ratio de la densidad energética promedio en el detector en dos simulaciones: con ($E_1$) y sin ($E_2$) aire turbulento}
    \label{fig:comparacion_2d}
\end{figure}

Es posible también encontrar la amplitud máxima en ambas simulaciones para comparar con \ref{ec:log_amplitud}. Esto se puede hacer simplemente encontrando el máximo de la magnitud $|E_1|$ que llega al detector. Para comparar de forma razonable, se encontró la amplitud para diferentes escalas de turbulencia, repitiendo varias veces para cada escala. De esta forma se encontró un promedio del cuadrado del logaritmo. El resultado se puede ver en la figura \ref{fig:log_amlitude1}, con una versión normalizada al valor máximo en la figura \ref{fig:log_amlitude2}. Se realizó un promedio de 10 simulaciones para cada escala de turbulencia, tomándose una rejilla de 200 por 100 puntos.

\begin{figure}
    \centering
    \includegraphics[width=0.7\linewidth]{figures/log-amplitude.png}
    \caption{Promedio del logaritmo-amplitud según la escala de la turbulencia, medida en unidades de celda de simulación. La escala de la turbulencia es la magnitud de las inhomogeneidades descritas anteriormente}
    \label{fig:log_amlitude1}
\end{figure}

\begin{figure}
    \centering
    \includegraphics[width=0.7\linewidth]{figures/log-amplitude_good.png}
    \caption{Promedio normalizado del logaritmo-amplitud según la escala de la turbulencia, medida en unidades de celda de simulación. La función $f(l) \propto l^{-7/3}$ sirve como comparación con \ref{ec:log_amplitud}, para lo cual se ha normalizado dividiendo por el valor máximo del logaritmo-amplitud. }
    \label{fig:log_amlitude2}
\end{figure}

Se ha experimentado también con obtener la permitividad a través de las ecuaciones para el índice de refracción dadas en \cite{ciddor_refractive_1996}. En este caso, se han tomado la temperatura y la humedad como campos aleatorios definidos por una distribución gaussiana. Después, asumiendo que el aire es un gas ideal, se ha utilizado la ley de los gases ideales para obtener la presión. Con los datos de longitud de onda, temperatura, presión y humedad se ha encontrado la permitividad. En vez de realizar una interpolación para suavizar el resultado, se exploró realizar una convolución con un kernel constante y gaussiano. Para una temperatura de $T=30\pm7$ y una humedad de $h = 0.3\pm0.05$, junto con un kernel gaussiano de dimensiones 10 celdas por 10 celdas, la permitividad resultante puede verse en \ref{fig:perm_aleatoria_ciddor}. El principal desafío al utilizar este método es mantener las propiedades estadísticas de los campos aleatorios, ya que al realizar convoluciones tanto la media como la desviación estándar tienden a reducirse. Las pruevas preliminares realizadas indican que estas propiedades estadísticas se reducen proporcionalmente a las dimensiones del kernel. 

\begin{figure}
    \centering
    \includegraphics[width=0.7\linewidth]{figures/perm_aletoria_ciddor.png}
    \caption{Utilizando el modelo dado en \cite{ciddor_refractive_1996}, es posible calcular la permitividad como un campo aleatorio a partir de la temperatura y la humedad. Se a aplicado una convolución para suavizar el resultado.}
    \label{fig:perm_aleatoria_ciddor}
\end{figure}


\section{Conclusiones}
\subsection{Conclusiones del trabajo}

Este TFG a explorado el impacto de la turbulencia en ondas infrarrojas. Se han visto efectos de atenuación y dispersión que afectan la calidad de la señal. Los fundamentos teóricos, basados en las ecuaciones de Maxwell y las funciones de Green, proporcionan un marco robusto para modelar este fenómeno. Las simulaciones FDTD revelan que la escala de la turbulencia es un factor crítico, con mayores inhomogeneidades resultando en una mayor pérdida de energía.

Los resultados tienen implicaciones prácticas para tecnologías que dependen de ondas IR, como sistemas de comunicación óptica y sensores remotos. Comprender y mitigar los efectos de la turbulencia puede mejorar la fiabilidad de estos sistemas, especialmente en entornos con condiciones atmosféricas adversas. 

\subsection{Futuras líneas de trabajo}

Una posible futura linea de trabajo es extender el análisis numérico presentado a tres dimensiones. Debido a la forma en la que se maneja internamente la simulación en la librería utilizada, esta extensión debería resultar sencilla. El subsecuente análisis y representación de los resultados obtenidos podría plantear retos interesantes. 

Cabe también explorar más profundamente la unión entre la simulación electromagnética y la del fluido. Esto sería clave si se quiere investigar efectos en los que la onda electromagnética afecta al fluido, por ejemplo efectos de disipación energética de la onda que alteran la energía del fluido. Para esto podría llegar a ser necesario una implementación propia del método FDTD.

Una aproximación más cuidadosa a la generación de la permitividad aleatoria, como las usadas en \cite{moss_finite-difference_2002} podría también mejorar la simulación. Podría también comenzarse este análisis estableciendo un método riguroso para generar campos aleatorios continuos con propiedades estadísticas dadas a partir de convoluciones. 

Otras futuras lineas pueden incluir: Validar los resultados de la simulación con datos experimentales reales de propagación de IR en entornos turbulentos, explorar estrategias de mitigación directamente en el entorno de simulación (como la implementación de ópticas adaptativas), investigar turbulencias no isotrópicas (como flujo turbulento).


\begin{otherlanguage}{spanish}
    %\printbibliography[heading=bibintoc, title={Referencias}]
    \printbibliography[heading=bibnumbered, title={Referencias}]
\end{otherlanguage}

\section{Anexos}
% Incluir algunas funciones con https://www.overleaf.com/learn/latex/Code_listing#Using_listings_to_highlight_code
% Es posible importar de un archivo de python

El directorio en su totalidad donde se encuentra el código de Python junto con las figuras se puede encontrar en el siguiente repositorio: \href{https://github.com/villarjorge/tfg}{github.com/villarjorge/tfg}. Aquí se incluirá parte de los scripts que han resultado más útiles. Este repositorio contiene código de otros repositorios, en específico \href{https://github.com/barbagroup/CFDPython}{"12 steps to Navier-Stokes"} una iniciativa detallada en \cite{barba_cfd_2018}; junto con la base de datos \cite{polyanskiy_refractiveindexinfo_2024}, que contiene código para simular el índice de refracción.

\lstset{style=mystyle}

\begin{otherlanguage}{spanish}
    \lstinputlisting[language=Python, caption=Script que contiene funciones para gestionar la simulación]{code/helpers.py}
\end{otherlanguage}

\end{document}
